\section{Dust Summary}

\subsection{Australia}

The Australian dust story is one of extremes, convolving the ENSO cycle with flooding, droughts, and winds.
Fluvial deposits in the Eyre basin are required to provide a dust base, which will only have mobility during suitable dry weather conditions, which can only be saltated and transported by large winds(todo cites).

Australia is the largest source of dust in the southern hemisphere. 
Aeolian dust has many direct effects including reduced surface insolation, mineral transfer to remote ocean regions, and health degradation in populated areas.
Direct and indirect effects of dust have many implications which are not fully understood, with many models still struggling to explain the role of dust at larger scales \cite{Rotstayn_2011}.

Synoptic scale measurements of dust plumes in Australia are kept by the BoM and can be used to estimate dust transport caused by large storms(\cite{Leys_2011}, todo more cites here). 
These estimates exemplify the large variability in Australian annual dust transport with single storms moving up to 2.5 Mt of dust in a single day, a significant portion of the yearly mean re\cite{Leys_2011,Shao_2007}.

\subsection{Oceanic Deposition}
Oceanic carbon cycling is a complex system in which iron is a limiting factor, required by diazotrophs in order to fix atmospheric Nitrogen into a more bioavailable form such as ammonia. 
Atmospheric deposition into the oceans is a very poorly constrained variable in global models (\cite{Grand_2015} more CITES). 
Various estimates of trace element oceanic depositions are required due to a dearth of in situ measurements in remote open ocean regions.

Measurements of dissolved iron (DFe) at very low concentrations like those found in surface ocean waters are very easily contaminated, which has contributed the the fragmentary and scarce nature of DFe ocean data sets \cite{Rijkenberg_2014}.
Some headway has been made with the recent GEOTRACES program which has several transects of the major oceans and measures trace elements over multiple depths including Al, Ba, Cu, Cd, Fe, Mn, Ni, Pb, and Zn.

Typically three methods are used to estimate oceanic deposition: a) land based extrapolation, b) shipboard aerosol measurements, and c) surface water measurements. 
These methods each have their own problems, and all suffer from high uncertainties. 
Recent analysis of the US Climate Variability and Predictability (CLIVAR)-CO$_2$ Repeat Hydrography Program (using method c) predicted total deposition flux with uncertainty at a factor of 3.5 due mostly to assumed Aluminium solubility and residence time \cite{Grand_2015}.
A fourth method using Aerosol Optical Depth(AOD) recorded by satellites in combination with estimated Fine Mode Fraction(FMF), wind, and precipitation data allows remote ocean deposition estimates but this method lacks good verification(again due to lack of in situ measurements). 


  
  
  
  
  
  
  
  
  
  