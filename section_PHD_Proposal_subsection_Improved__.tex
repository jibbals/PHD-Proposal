\section{PHD Proposal}

\subsection{Improved Modelling resolution}
Higher resolution modelling over Australia will allow for improved validation with in situ measurements, as well as regional and city scale simulation analysis. 
Dust simulation in particular can be greatly improved with finer surface wind resolution and will be one of the areas I will focus on with validation against both in situ and satellite measurements of AOD and dust concentrations across Australia.
One of my goals for my PHD will be to improve the GEOS-Chem model by setting up meteorological fields with higher horizontal resolution over Australia and NZ.
This has been done already over China, North America, and Europe (\cite{Chen_2009}, 2 more cites needed) but not for anywhere in the southern hemisphere.

\subsection{Ozone intrusions}
Ozone in the lower atmosphere is a serious hazard that causes health problems \cite{Hsieh_2013}, damages agricultural crops worth billions of dollars [2], and increases the rate of climate warming[3]



2:Avnery, S., et al., Global crop yield reductions due to surface ozone exposure: 1. Year 2000 crop production losses and economic damage. Atmos. Env., 45: 2284-2296, 2011
3:Myhre, G. and Shindell, D., Chapter 8: Anthropogenic and Natural Radiative Forcing, in Climate Change 2013: The Physical Science
Basis. Working Group 1 Contribution to the Fifth Assessment Report of the Intergovernmental Panel on Climate Change, 2013.

\subsection{Isoprene and VOCs}
Another source(the main source) of ozone in the lower troposphere is chemical formation following emissions of precursor gases, including volatile organic compounds (VOCs) like isoprene, and nitrogen oxides (NOx).
