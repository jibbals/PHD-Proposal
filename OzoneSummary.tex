\subsection{Ozone and Isoprene Summary}
Ozone in the lower atmosphere is a serious hazard that causes health problems \cite{Hsieh_2013}, damages agricultural crops worth billions of dollar \cite{Avnery_2011}, and increases the rate of climate warming \cite{IPCC_2013_chap8}.
  
Photochemical oxidation of CO, CH4, and NMVOCs while NOx is present, as well as downward stratospheric transport, are the two main sources of surface ozone.
HCHO and NO2 regulate tropospheric oxidation capacity through O3 production, as well as being health hazards.
The HCHO/NO2 ratio can be used to determine whether surface O3 is NO2 or VOC limited \cite{Mahajan_2015}.
NO2 is a common pollutant in populated areas, released primarily by power generation and transport.


The main NMVOC is isoprene(\cite{Guenther_2006}) which is hard to directly measure, instead formaldehyde is often used as a proxy \cite{Marais_2012, bauwens2013satellite}. 
Satellites can use DOAS analysis methods on solar radiation absorption spectra to measure column HCHO.
Several satellite data sets are publicly available and have been used to show increasing HCHO trends in developing countries \cite{Mahajan_2015}.
  