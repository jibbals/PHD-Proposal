\subsection{Improved GEOS-Chem Modelling resolution}
Higher resolution modelling over Australia will allow for both improved validation with in situ measurements, and regional and city scale simulation analysis. 
Dust simulation in particular can be greatly improved with finer surface wind resolution and will be one of the areas I will focus on with validation against both in situ and satellite measurements of AOD and dust concentrations across Australia.
One of my goals for my PHD will be to improve the GEOS-Chem model by setting up meteorological fields with higher horizontal resolution over Australia and NZ.
This has been done already over China \cite{Chen_2009}, North America, and Europe ( 2 more cites needed) but not for anywhere in the southern hemisphere.

First sufficient disk space will be required to hold several years worth of high resolution meteorological data and emissions, this is expected to be to the order of 100s of gigabytes.
After determining the desired spatial and temporal boundaries for improved resolution, raw data will be downloaded, verified, and re-gridded using partially written code from Bob Yantosca.
Finally emissions data needs to be reworked over the domain, ensuring emissions factors and chemistry is maintained.
This last requirement may be eased by the new HEMCO module implemented within the ESMF environment (compatible with GEOS-Chem v10) which recalculates emissions on any user specified grid.



\subsection{Isoprene and VOCs model validation over Australia}
Another source(the main source) of ozone in the lower troposphere is chemical formation following emissions of precursor gases, including volatile organic compounds (VOCs) like isoprene, and nitrogen oxides (NOx).
Estimates put isoprene emission at roughly 550 Tg/yr \cite{Guenther_2006, Monks_2014}, emitted by trees and shrubs mostly during the day.

In Australia, biogenic ozone precursor sources are highly uncertain, impeding accurate ozone modelling and projections. These uncertainties could explain why models of formaldehyde (HCHO) over Australia are not good at reproducing observed data \cite{Stavrakou_2009}. Atmospheric HCHO data exists over Wollongong, and some campaigns have taken place elsewhere in Australia, however these do not give a full overview of the continent's emissions. Satellite-based observations do exist with good coverage but these have large uncertainties. 

Determination of purely biogenic emissions from satellite-based observations is possible by filtering out biomass burnings using the fire counts and aerosol absorption optical depth(AAOD). The AATSR, Aqua, and Terra satellites have fire counts which can be used, and the EOS Aura satellite contains the AAOD measurements. These data can effectively filter satellite HCHO measurements in order to compare with biogenic models and this process has been implemented already in a similar way over South Africa \cite{Marais_2012}.

HCHO is an intermediate chemical product and can be used as a proxy for determining biogenic isoprene emissions. 
Isoprene is commonly measured by proxy through satellite HCHO vertical column densities (VCDs). HCHO, once filtered for fire plumes and anthropogenic influences, can determine isoprene emission through the use of a chemical transport model (CTM).
This method of inference has been used successfully in several countries including North America\cite{Palmer_2003}, South America\cite{Barkley_2013}, and Africa \cite{Marais_2012}.

Satellite HCHO measurements exist since 2004 (OMI) and 2006 (GOME2), and the ground based measurements over Wollongong exist since 1996. While the ground based data is spatially sparse it can be used as validation of models and satellite data. Combining the long record of measurements from Wollongong with the decadal-scale satellite data and chemical transport modelling in this PhD project will allow the first large-scale quantification of isoprene measurements in Australia, paving the way for more accurate ozone projections.

\subsection{Quantifying stratosphere to troposphere ozone intrusions}
When looking at a vertical profile of ozone the effect of STT is clear, with the right filter we can quantitatively observe and characterise STTs.
Using a Fourier filter on the (TODO DESCRIBE DATA AND OWNERS) will give us the first southern hemispheric analysis of STTs.


\subsection{Australian model dust validation}
Notes from Paper and validation of GOES dust simulation?
  
Dust emission is considered to be dependent on surface wind speeds to the third order \citep{Duncan_Fairlie_2007}, and this introduces problems when the large horizontal grid does not capture the sub grid wind speed distribution \cite{Ridley_2013}.
  
  
  
  
  
  
  