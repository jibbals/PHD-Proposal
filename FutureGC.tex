\section{Future schedule}

\subsection{2016}

Running a GEOS-Chem simulation and storing available VOCs and comparing with the available in-situ measurements as well as satellite data over Australia will be the primary in 2016.
The methods for comparison are mathematically complex in order to account for various reflectances and extinction effects as well as instrument sensitivities and understanding and utilising these concepts will take some time.
The results from this analysis will contribute to planning for a field campaign to take place in south eastern Australia in 2017 focused on biogenics as well as a possible aircraft campaign in 2018 led by the National Center for Atmospheric Research (NCAR).
Results analysis and presentation will become the focus once the methods for comparison and anthropogenic filtering are completed.

Secondly I will be working on improving the GEOS-Chem model to allow higher resolution simulations over Australia and New Zealand.
The new resolution will need to be tested and compared against benchmark simulations in order to verify that the implementation has not caused any undesired effects.
This will involve collaboration with several people involved in GEOS-Chem development and the National Environmental Science Programme (NESP) clean air and urban landscapes hub.

Lastly I will be finalising ozone analysis from the in-situ ozonesonde measurements and looking at GEOS-Chem ozone profiles in the same areas, as well as finalising Australian dust analysis and looking for improvements made by running at the finer resolution.

