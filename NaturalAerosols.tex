\subsection{Natural gas and aerosol emissions in Australia}

In this Thesis I will combine satellite and ground based atmospheric measurements with modelled emissions estimates to clarify the impact of Australian biogenic gases and aerosols on our health and atmosphere.

Natural emissions in Australia are important as areas from where trace gases and aerosols are naturally emitted form the majority of our continent.
These biogenics affect surface pollution levels and can alter the radiative and particulate matter distribution of the atmosphere with harmful results.
For example ozone in the lower atmosphere is a serious hazard that causes health problems \cite{Hsieh_2013}, damages agricultural crops worth billions of dollar \cite{Avnery_2011}, and increases the rate of climate warming \cite{IPCC_2013_chap8}. 

The Australian outback is described by extremely large and diverse environments. 
The outback is considered to be everywhere outside of small to large cities, normally with populations of less than 100 000 people.
Uncertainty in estimates of atmospheric gas and particulate densities and distributions stem from a lack of in-situ measurements.
\citet{Guenther_2006} shows how the Australian outback is among the worlds strongest isoprene sources with large forests near coastal cities emitting greater than 16 mg m$^{-2}$ h$^{-1}$ (see figure \ref{fig:meganisoprene}).

  