\section{Dust}

\subsection{McTainsh 2005}
4.2 Mt dust load estimated for October 2002 dust storm.

\subsection{Red Dawn: \cite{Leys_2011}}
PM10 concentrations and mass transport during "Red Dawn" - Sydney 23 September 2009.

Worst dust storm to hit Sydney since records began in 1940, with heavy dust for nine hours and visibility reduced to 400m at the airport.
Storms are common in the arid inland but rare on the east coast.
Plume dust concentration generally calculated either by simulation, or by plume volume * aerosol concentration.
$$M = AhC$$ 
where A is plume area, h is plume height, C is g/m3 of dust, M is dust load in grams.
These measurements are hampered by coarse temporal resolution of at best three hourly intervals of the Bureau of Meteorology(BoM)'s synoptic visibility observations.

Dust concentrations sourced from some stations of the DECCW AQMN.

Visibility is used to estimate total suspended particle(TSP) concentration, here a conservative predictor of PM10 c for given V is used from (Zhangbei TODO CITE).

PM10 dust transport is defined and measured with plume dust concentrations measured at various altitudes with a kite. 

Dust storm preceded by 9 dry years, 13 hot years, and was itself the hottest year on record.

PM10 normally $10{\mu}g/m^3$ in Sydney peaked at $15366{\mu}g/m^3$.
\~2.54 Mt of PM10 dust transported off shore.
Storm on par with large Chinese dust storm.
Dust ceiling around 2500m, up to 3000m.



