\subsubsection{Going Forward}
In Australia, biogenic ozone precursor sources are highly uncertain, impeding accurate ozone modelling and projections.
Atmospheric HCHO data exists over Wollongong from a Fourier transform spectrometer in operation since 1996, and some Australian measurements exist from campaigns such as the Measurements of Urban, Marine and Biogenic Air (MUMBA).
However these do not give a full overview of the continent's emissions. Satellite-based observations cover the entire continent, although with larger uncertainties involved due to horizontal resolution, cloud cover, and various other factors. 

Determination of purely biogenic emissions from satellite-based observations is possible by filtering out biomass burning using the fire counts and AOD.
The AATSR, Aqua, and Terra satellites have fire counts which can be used to determine when HCHO is caused by burnings or smoke plumes, as well as accounting for anthropogenic emissions (gas flaring).
These data can effectively filter satellite HCHO measurements in order to compare with biogenic models and this process has been implemented already in a similar way over Western Africa \cite{Marais_2012}.

Isoprene is commonly measured by proxy through satellite HCHO vertical column densities (VCDs). HCHO, once filtered for fire plumes and anthropogenic influences, can determine isoprene emission through the use of a chemical transport model (CTM).
This method of inference has been used successfully in several countries including North America\cite{Palmer_2003}, South America\cite{Barkley_2013}, and Africa \cite{Marais_2012}.
Using the same methodology to retrieve estimates of isoprene emissions over Australia will be one of the focuses of this thesis.

Satellite HCHO measurements exist since 2004 (OMI) and 2006 (GOME2), and the ground based measurements over Wollongong exist since 1996. While the ground based data is spatially sparse it can be used as partial validation of models and satellite data. 
Combining the long record of measurements from Wollongong with the decadal-scale satellite data and chemical transport modelling in this PhD project will allow the first large-scale quantification of isoprene measurements in Australia, paving the way for more accurate ozone projections.
