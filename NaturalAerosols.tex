\subsection{Natural gas and aerosol emissions in Australia}

Australia is largely covered by environments which are not heavily influenced by human activity.
These environments are sources of naturally released trace gases which make up less than 1\% of earth's atmosphere.
Naturally occurring trace gases in the atmosphere can have a large impact on living conditions.
They react in complex ways with other elements (anthropogenic and natural), as well as affecting various ecosystems upon which life depends.
Natural emissions affect surface pollution levels and can alter the radiative and particulate matter distribution of the atmosphere with harmful results.
For example, ozone (O$_3$) in the lower atmosphere is a serious hazard that causes health problems \cite{Hsieh_2013}, damages agricultural crops worth billions of dollars \cite{Avnery_2011}, and increases the rate of climate warming \cite{IPCC_2013_chap8}.
Particulate matter in the atmosphere is also a major problem, causing an estimated 2-3 million deaths annually \cite{Hoek_2013, 19627030, Silva_2013, Lelieveld_2015}.

The Australian outback includes extremely large and diverse environments.
Much of the landscape outside of urban areas is undeveloped and sparsely inhabited.
In Australia most long term air quality measurements are performed in or near large cities.
however estimates of atmospheric gas and particulate densities and distributions over much of the rest of the continent are uncertain and lack in-situ measurements.
%For instance the Total Carbon Column Observing Network (TCCON) has sites at Darwin and Wollongong, and the Aerosols Robotic Network (AERONET) 