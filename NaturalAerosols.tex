\subsection{Natural gas and aerosol emissions in Australia}
Natural emissions in Australia are important due to the large population sparse areas from where dust, isoprene, and other important biogenics are released.
The Australian outback is described by extremely large and diverse environments, mixed with a dearth of in-situ measurements this leads to uncertainty in estimates of atmospheric gas and particulate densities and distributions.
(TODO: BVOC emissions from MEGAN or elsewhere examples)

One source of information which covers the entirety of Australia is remote sensing performed from satellites which overpass daily and record reflected solar spectra.
These can be used to garner details of several chemicals as well as estimates of their distribution in vertical columns over the land.
While satellite data is very good for covering huge areas (the entire earth) it only exists at a particular time of day, is subject to cloud cover, and generally does not have fine horizontal or vertical resolution.
A problem with the vertical profiles from satellites is that they are estimates based on forward radiative transfer modelling.
This leaves us with a best guess for chemical concentrations at any altitude which may need verification or adjustment.


(TODO: ROADMAP OF SATELLITE + MODEL quantification of natural emissions)



  
  
  