\subsection{Isoprene and formaldehyde(HCHO)}

Biogenic isoprene emissions are far greater than anthropogenic emissions of VOCs \cite{Guenther_2006}. 
The estimates for isoprene emissions lack verification especially over Australia where very little work has been done to measure biogenic VOCs (BVOCs) like isoprene and monoterpenes.
This lack of accuracy in BVOC emissions has a large effect on determining with confidence the sources and distribution of pollutants including ozone and other greenhouse gases (GHGs).
It has been shown that even the diurnal pattern of isoprene emissions has an effect on modelling ground level ozone \cite{Hewitt_2011}.

The main NMVOC is isoprene \cite{Guenther_2006} which is hard to directly measure, instead formaldehyde is often used as a proxy \cite{Marais_2012,bauwens2013satellite}.
Satellites can use DOAS techniques with radiative transfer calculations on solar radiation absorption spectra to measure column HCHO.
Several satellite data sets are publicly available and have been used to show increasing HCHO trends in developing countries \cite{Mahajan_2015}.

TODO What does HCHO and Isoprene look like in Australia !!!