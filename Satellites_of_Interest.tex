\subsection{Satellites}
\label{satellites}
Several satellites provide long term trace gas observations with near complete global coverage, including the ERS-2 launched in April 1995 which houses the GOME ultraviolet and visible (UV-Vis) spectrometer, the AURA launched in July 2004 which houses the OMI UV-Vis spectrometer, the MetOp-A and B launched in October 2006 and September 2012 respectively both housing a GOME-2 UV-Vis spectrometer.
These satellites are on Low Earth Orbit (LEO) trajectories and overpass any area up to once per day. 
They record near nadir reflected spectra between around 250-700~nm split into spectral components at around $0.3$~nm in order to calculate trace gases including O$_3$, NO$_2$, and HCHO.
An example of a spectrum retrieved from the GOME-2 instrument is given in figure \ref{fig:gomeproducts}.

Formaldehyde (HCHO) is often used as a proxy to estimate isoprene emissions \cite{Marais_2012,bauwens2013satellite}.
Satellites can use DOAS techniques with radiative transfer calculations on solar radiation absorption spectra to measure column HCHO (eg: \citet{Leue_2001}).
Several public data servers are available which include products from the satellites just mentioned, including NASA's Mirador (http://mirador.gsfc.nasa.gov/) and the Belgian Institute for Space Aeronomy (IASB-BIRA) Aeronomy (http://h2co.aeronomie.be/).

DOAS techniques with radiative transfer calculations use various parts of a solar radiation absorption spectra to measure trace gases through paths of light.
DOAS methods can be heavily influenced by the initial estimates of a trace gas profile (the a priori) which is often produced by modelling, so when comparing models of these trace gases to satellite measurements extra care needs to be taken to avoid introducing bias from unrealistic a priori assumptions.
A way to remove these a priori influences in order to compare models and satellites is through the satellite's averaging kernal, which is a measure of the sensitivity of the instrument to the trace gas's radiance at various heights multiplied by the sensitivity of the DOAS technique's forward radiative transfer model (RTM) to the amount of trace gas at various heights near the a priori \cite{Eskes_2003}.

The RTM used in DOAS techniques is based on Beer's law relating the attenuation of light to the properties of the medium it travels through.
Beer's law states that $ T = I/I_0 = e^{-\tau} $ with T being transmittance, $\tau$ being optical depth, and I, I$_0$ being radiant flux received at instrument and emitted at source respectively.
Using 
$ \tau_i = \int \rho_i \beta_i ds $ gives us:
$$ I = I_0 \exp {\left( \Sigma_i \int \rho_i \beta_i ds \right) } $$
Where i represents a chemical species index, $\rho$ is a species density(molecules per cm$^3$), $\beta$ is the scattering and absorption cross section area (cm$^2$), and the integral over ds represents integration over the path from light source to instrument.
The forward RTM used for satellite data products also involves functions representing extinction from Mie and Rayleigh scattering, and the efficiency of these on intensities from the trace gas under inspection, as well as accounting for various atmospheric parameters which may or may not be estimated (e.g. albedo).
Rayleigh and Mie scattering describe two kinds of particle effects on radiation passing through a medium. Rayleigh scattering is heavily wavelength dependent, and is the dominant form of scattering from particles up to roughly one tenth of the wavelength of the light. Mie scattering is more dominant from larger particles, and has less wavelength dependence.

To convert the trace gas profile from a reflected solar radiance column (slanted along the light path) into a purely vertical column requires calculations of an air mass factor (AMF) due to the non strict path of light measured by the instrument.
The AMF is normally a scalar value for each horizontal grid point which will equal the ratio of the total vertical column density to the total slant column density. This value should also account for instrument sensitivities to various wavelengths at various altitudes, and is unique for each trace gas under consideration.

Instruments including MODIS on board the AQUA and TERRA satellites are able to determine aerosol optical depth (AOD), a measure of atmospheric scatter and adsorption. 
An AOD of under 0.05 indicates a clear sky, while values of 1 or greater indicate increasingly hazy conditions.
This is an important atmospheric property allowing us to track dust storms and pollution events as well as determine where measurements from other instruments may be compromised by high AOD.
Satellite measured AOD requires validation by more accurate ground based instruments like those of AERONET which uses more than 200 sun photometers scattered globally. 