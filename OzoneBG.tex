\subsection{Tropospheric ozone}

Ozone in the lower atmosphere is a serious hazard that causes health problems \cite{Hsieh_2013}, damages agricultural crops worth billions of dollar \cite{Avnery_2011}, and increases the rate of climate warming \cite{IPCC_2013_chap8}. 
Ozone and is one of the six substances monitored by the Ambient Air Quality (AAQ) National Environment Protection Measure (NEPM), which is the Australian framework for air quality measurement and reporting aiming for \begin{quote}"adequate protection of human health and well-being" .\end{quote}
NEPM AAQ benchmarks are listed in figure \ref{fig:nepmtable}

Photochemical oxidation of CO, CH4, and other non-methane volatile organic chemicals (NMVOCs) while NOx is present, as well as downward stratospheric transport, are the two main sources of surface ozone.
HCHO and NO2 regulate tropospheric oxidation capacity through O3 production, as well as being health hazards.
The HCHO/NO2 ratio can be used to determine whether surface O3 is NO2 or VOC limited \cite{Mahajan_2015}.
NO2 is a common pollutant in populated areas, released primarily by power generation and transport. 

Ozone-rich air mixes irreversibly down from the stratosphere during meteorologically conducive conditions \citep{Sprenger2003,Mihalikova2012}; these are referred to as Statosphere to Troposphere Transport events (STTs).
In the extra-tropics, STTs most commonly occur during synoptic-scale tropopause folds \citep{Sprenger2003} and are characterised by tongues of high Potential Vorticity (PV) air descending to low altitudes.
These tongues become elongated and filaments separate from the tongue which mix into tropospheric air.
Stratospheric ozone brought deeper (lower) into the troposphere is more likely to affect the surface ozone budget and tropospheric chemistry \citep{Zanis2003,Langford_2009}.

Ozone transported in from the stratosphere is not the primary source, however climate change may drastically increase this quantity \cite{Hegglin_2009}.
The amount of ozone from STTs and photochemical production are estimated to be around 550 Tgyr-1 and 5100 Tgyr-1, respectively \cite{Stevenson_2006}. 
The main ozone removal processes are chemical destruction and wet deposition.
  
\textbf{Ozone and PM deaths better words pls}

There is evidence for chronic effects on mortality through several large cohort studies for PM2.5 (Hoek et al 2002, Krewski et al 2009, Lepeule et al 2012), while evidence for chronic effects of ozone derives mainly from one study (Jerrett et al 2009). 

\begin{quote}
Anenberg et al (2010) used output from a global atmospheric model to estimate 3.7 $\pm$ 1.0 million deaths annually due to anthropogenic (present-day relative to preindustrial) changes in PM2.5 and 0.7 $\pm$ 0.3 million due to ozone. Brauer et al (2012) used high-resolution satellite observations of PM2.5 together with a global atmospheric model and an extensive compilation of surface measurements to better represent global air pollution exposure. These exposure estimates were then used to estimate 3.2 $\pm$ 0.4 million premature deaths due to PM2.5 and 150000 (50000 to 270000) due to ozone (Lim et al 2012).
\end{quote} \cite{Silva_2013}

Silva 2013 uses a model ensemble from ACCMIP of 14 models comparing aerosol loads from 1850(preindustrial) to 2000. Using methods as in Anenberg et al 2010 they determine premature mortality and with Monte Carlo methods determine the CIs.

Yearly global deaths from ozone, PM2.5:
472000(149/million), 2110000(665/million) \cite{Silva_2013}

((tenuous?)LINK TO ISOPRENE)

One dominant precursor to ozone is isoprene, which in the atmosphere reacts rapidly with hydroxyl radicals to form peroxy radicals, which can react with nitrogen oxides to form ground-level ozone .
(TODO - Chemistry pathway here)
  