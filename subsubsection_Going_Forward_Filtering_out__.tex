\subsubsection{Going Forward}

Filtering out the non biogenic sources of HCHO over Australia will be performed similarly to \cite{Marais_2012}, modified to suit Australia.

Determination of purely biogenic emissions of HCHO from satellite-based observations is possible by filtering out biomass burning using the fire counts and AOD.
The AATSR, Aqua, and Terra satellites have fire counts which can be used to determine when HCHO is caused by burning or smoke plumes, as well as accounting for other anthropogenic emissions (gas flaring).

Applying the formulas laid out in section \ref{proxy} will produce estimates of isoprene emissions using HCHO as a proxy.
The effect of uncertainties due factors like smearing and a priori measurement uncertainty on the quality of isoprene emissions will then be determined.

Finally isoprene emissions estimates can be compared to in situ measurements, such as the Measurements of Urban, Marine and Biogenic Air (MUMBA).

This method of inference has been used successfully in several countries including North America\cite{Palmer_2003}, South America \cite{Barkley_2013}, and Western Africa \cite{Marais_2012}.

The outcomes from the emissions analysis will contribute to planning for the COALA aircraft campaign which may take place in 2018.
