\subsection{Natural gas and aerosol emissions in Australia}

In this Thesis I will combine satellite and ground based atmospheric measurements with modelled emissions estimates to clarify the impact of Australian biogenic gases and aerosols on our health and atmosphere.

Areas from where trace gases and aerosols are naturally emitted form the majority of Australia.
Naturally occurring trace elements in the atmosphere can have a large impact on living conditions and will react in complex ways with anthropogenic emissions as well as affect various ecosystems upon which we depend.
These biogenics affect surface pollution levels and can alter the radiative and particulate matter distribution of the atmosphere with harmful results.
For example ozone in the lower atmosphere is a serious hazard that causes health problems \cite{Hsieh_2013}, damages agricultural crops worth billions of dollar \cite{Avnery_2011}, and increases the rate of climate warming \cite{IPCC_2013_chap8}.

The Australian outback is described by extremely large and diverse environments, loosely defined to be everywhere outside of small to large cities, normally with populations of less than 100 000 people.
In Australia most air quality measurements are performed in cities and trace gas readings are influenced greatly by anthropogenic influences.
However estimates of atmospheric gas and particulate densities and distributions over the continent are uncertain stemming from a lack of in-situ measurements.
\citet{Guenther_2006} shows how the Australian outback is among the worlds strongest isoprene sources with large forests near coastal cities emitting greater than 16 mg m$^{-2}$ h$^{-1}$ (see figure \ref{fig:meganisoprene}).

  