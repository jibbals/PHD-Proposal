One source of information which covers the entirety of Australia is remote sensing performed from satellites which overpass daily and record reflected solar spectra.
These can be used to garner details of several chemicals as well as estimates of their distribution in vertical columns over the land.
While satellite data is very good for covering huge areas (the entire earth) it only exists at a particular time of day, is subject to cloud cover, and generally does not have fine horizontal or vertical resolution.
Additionally vertical profiles from satellites are estimates based on forward radiative transfer modelling which requires good in-situ data to reduce inaccuracy.
This leaves us with a best guess for chemical concentrations at any altitude which may need verification or adjustment.

The existence of satellite data covering remote areas provides researchers an opportunity to analyse truly global information and put together a better idea of global climate and chemistry.
Natural chemical emissions from areas with no anthropogenic influence and no ground based measurements make up the majority of our continent \cite{VanDerA_2008} and require detailed analysis to inform national policy on air pollution levels (TODO:Cite nepm paper?).   
Comparing global models of atmospheric chemistry with satellite data to test our understanding of large scale natural chemistry and potentially improve those models, inform policy, and predict harmful events is a worthwhile goal.

In this thesis I will combine satellite and ground based atmospheric measurements with modelled emissions estimates to clarify the impact of Australian biogenic gases and aerosols on our health and atmosphere.