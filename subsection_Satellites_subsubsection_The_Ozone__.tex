\subsection{Satellites}

\subsubsection{The Ozone Monitoring Instrument (OMI)}
OMI is on board the NASA AURA Launched in 2004, contributed by Netherlands' Agency for Aerospace Programs (NIVR) working with the Finnish Meteorological Institute (FMI) and NASA. OMI views 740 wavelength bands covering the 264-504 nm spectral region.
The AURA satellite orbits at 705 km altitude in sun-synchronous polar orbit with a local equator crossing time of 13:45 on the ascending node.
OMI is a wide angle non-scanning and nadir-viewing instrument with a 115$\^{\circ}$ wide field of view collecting 2600~km backscattered irradiance swathes. spatial resolution of 13 by 24 km$^2$ at nadir. 

OMI global coverage occurs daily in typically 14 orbits, producing many data products available free to the public.
Global gridded products include surface spectral irradiance, ozone columns(DOAS and TOMS methods), Aerosol(near-UV and multi-wavelength algorithms), cloud fraction and Pressure, ozone profile, formaldehyde, bromine, and nitrogen dioxide.
(http://disc.sci.gsfc.nasa.gov/Aura/data-holdings/OMI) cite?

\subsubsection{Global Ozone Monitoring Experiment (GOME)}

GOME is a nadir-scanning spectrometer on board the ERS-2, launched in 1995.
Spatial resolution can be varied from 40 by 40 km to 40 by 320 km.
light is split into spectral components from 240 to 790 nm at 0.2 to 0.4 nm resolution.

Available gridded products include Ozone, NO2, BrO, SO2, H2O, HCHO, OClO, and cloud properties fully from 1996 - 2003. The instrument has degraded and limited data is available beyond 2003. 

\subsubsection{GOME-2}
(http://atmos.eoc.dlr.de/gome2/docs/DLR_GOME_PUM.pdf) cite?
GOME-2 was launched in 2006 on board EUMETSAT's Meteorological Operational Satellite (MetOp-A). 
The instrument is a spectrometer with the same specifications as GOME, however spatial resolution can be varied from 5 by 40 km to 80 by 40~km.

Available gridded data products include O3, NO2, BrO, SO2, H2O, HCHO, OClO, CHOCHO, and cloud properties (fraction, albedo, optical thickness, height, pressure).
  