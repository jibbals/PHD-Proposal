\subsection{Improved GEOS-Chem Modelling resolution}

Higher resolution modelling over Australia will allow for both improved validation with in situ measurements, and regional scale simulation analysis. 
Dust simulation in particular can be greatly improved with finer surface wind resolution and will be one of the areas I will focus on with validation against both in situ and satellite measurements of AOD and dust concentrations across Australia.
One of my goals for my PHD will be to improve the GEOS-Chem model by setting up meteorological fields with higher horizontal resolution over Australia and NZ.
This has been done already over China \cite{Chen_2009}, North America, and Europe ( 2 more cites needed) but not for anywhere in the southern hemisphere.

First sufficient disk space will be required to hold several years worth of high resolution meteorological data and emissions, this is expected to be to the order of 100s of gigabytes.
After determining the desired spatial and temporal boundaries for improved resolution, raw data will be downloaded, verified, and re-gridded using partially written code from Bob Yantosca.
All of the data will need be compared to other emissions estimates in order to ensure that we have an acceptable inventory for Australian emissions.
Finally emissions data needs to be reworked over the domain, ensuring emissions factors and chemistry is maintained.
This last requirement may be eased by the new HEMCO module implemented within the ESMF environment (compatible with GEOS-Chem v10) which recalculates emissions on any user specified grid.
  
Since the release of v10-01, GEOS-Chem is grid-independent allowing any horizontal resolution to be run globally. 
It is also compatible with Earth System Modeling Framework (ESMF), allowing GEOS-Chem to run as the chemical module for Earth System Models (ESMs). 
GEOS-Chem allows distributed-memory parallelisation which can improve chemistry simulation performance almost linearly with increased cores (at least up to 240) \cite{Long_2015}.

For many chemical species GEOS-Chem can be run in a nested fashion, with high resolution in regional areas (0.25$^{\circ}$ x 0.3125$^{\circ}$ over North America, ,Asia, or Europe) while a lower global resolution is implemented allowing more reasonable run times.
This nested modelling requires some work in order to set up the emissions inventories but can greatly improve model accuracy.
A good example of this is the nested simulation of CO run by \citet{Yan_2014} which improved mean model bias against the HIPPO campaign from -9.2\% to 1.4\%, this means the modelled CO more closely matched the measurements.

Within atmospheric models emissions are generally determined by a combination of base emissions and multiplicative scaling factors. 
These scaling factors are used to account for temporal variability as well as allowing environmental parameterisations such as sub-grid surface wind speeds \cite{Ridley_2013,Zender_2003}.
A new module which facilitates nested runs is Harvard–NASA Emission Component (HEMCO), which can compile emissions inventories on the fly from any user-defined temporal or spatial resolution \cite{Keller_2014}.
This module allows fine scaling of emission factors on both regional and species bases.
HEMCO is used currently in various emissions inventories easily implemented by GEOS-Chem v10 including MEGAN, GFED-3, EDGAR, EMEP, and more \cite{Keller_2014}.