\subsection{Natural aerosol emissions in Australia}
Natural emissions in Australia are important due to the large population sparse areas from where dust, isoprene, and other important aerosols are released.
The Australian outback is described by extremely large and diverse environments, mixed with a dearth of in-situ measurements this leads to uncertainty in estimates of aerosol densities and distributions.

One source of information which covers the entirety of Australia is remote sensing performed from satellites which overpass daily and record reflected solar spectra.
These can be used to garner details of several chemicals as well as estimates of their distribution in vertical columns over the land.
While satellite data is very good for covering huge areas (the entire earth) it only exists at a particular time of day, is subject to cloud cover, and generally does not have fine horizontal or vertical resolution.
A problem with the vertical profiles from satellites is that they are estimates based on forward radiative transfer modelling.
This leaves us with a best guess for chemical concentrations at any altitude which may need verification or adjustment.

Several satellites provide long term trace gas data with near complete global coverage, including GOME (1, 2 metop A, and 2 metop B), OMI AURA, and SCIAMACHY.
These satellites are on Low Earth Orbit (LEO) trajectories and overpass any area up to once per day. 
They record near nadir reflected spectra between around 250-700~nm split into spectral components at around $0.3$~nm in order to calculate trace gases including O$_3$, NO$_2$, BrO, SO$_2$, and HCHO.
An example for the GOME-2A satellite is given in \ref{fig:gomeproducts}




  
  
  