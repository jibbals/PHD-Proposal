\subsection{Ozone, Isoprene, and other Biogenics Summary}
Ozone in the lower atmosphere is a serious hazard that causes health problems \cite{Hsieh_2013}, damages agricultural crops worth billions of dollar \cite{Avnery_2011}, and increases the rate of climate warming \cite{IPCC_2013_chap8}. 
Ozone and nitrogen dioxide are two of the six substances monitored by the Ambient Air Quality National Environment Protection Measure (NEPM). 
  
Photochemical oxidation of CO, CH4, and other non-methane volatile organic chemicals (NMVOCs) while NOx is present, as well as downward stratospheric transport, are the two main sources of surface ozone.
HCHO and NO2 regulate tropospheric oxidation capacity through O3 production, as well as being health hazards.
The HCHO/NO2 ratio can be used to determine whether surface O3 is NO2 or VOC limited \cite{Mahajan_2015}.
NO2 is a common pollutant in populated areas, released primarily by power generation and transport. 

Ozone transported in from the stratosphere is not the primary source, however climate change may drastically increase this source \cite{Hegglin_2009}.
There are two sources of tropospheric ozone: stratosphere troposphere transport (STT) and photochemical production, with estimated source fluxes of 550 Tgyr-1 and 5100 Tgyr-1, respectively \cite{Stevenson_2006}. 
Loss processes are chemical destruction and wet deposition.

The main NMVOC is isoprene \cite{Guenther_2006} which is hard to directly measure, instead formaldehyde is often used as a proxy \cite{Marais_2012,bauwens2013satellite}. 
Satellites can use DOAS analysis methods on solar radiation absorption spectra to measure column HCHO.
Several satellite data sets are publicly available and have been used to show increasing HCHO trends in developing countries \cite{Mahajan_2015}.
  
\subsection{Ozone and PM deaths}

There is evidence for chronic effects on mortality through several large cohort studies for PM2.5 (Hoek et al 2002, Krewski et al 2009, Lepeule et al 2012), while evidence for chronic effects of ozone derives mainly from one study (Jerrett et al 2009). 

\begin{quote}
Anenberg et al (2010) used output from a global atmospheric model to estimate 3.7 $\pm$ 1.0 million deaths annually due to anthropogenic (present-day relative to preindustrial) changes in PM2.5 and 0.7 $\pm$ 0.3 million due to ozone. Brauer et al (2012) used high-resolution satellite observations of PM2.5 together with a global atmospheric model and an extensive compilation of surface measurements to better represent global air pollution exposure. These exposure estimates were then used to estimate 3.2 $\pm$ 0.4 million premature deaths due to PM2.5 and 150000 (50000 to 270000) due to ozone (Lim et al 2012).
\end{quote} \cite{Silva_2013}

Silva 2013 uses a model ensemble from ACCMIP of 14 models comparing aerosol loads from 1850(preindustrial) to 2000. Using methods as in Anenberg et al 2010 they determine premature mortality and with Monte Carlo methods determine the CIs.

Yearly global deaths from ozone, PM2.5:
472000(149/million), 2110000(665/million) \cite{Silva_2013}

