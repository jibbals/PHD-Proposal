\subsection{Chemical Transport Models (CTMs)}
CTMs simulate production, loss, transport, and deposition of chemical species.
Generally CTMs will solve the continuity equations numerically for chemicals under inspection using one or both of the Eulerian (box) or Lagrangian (puff) frames of reference.
Where measurements exist they can be fed into CTMs as boundary and/or initial conditions in order to both examine real world concentrations and limit model uncertainty and divergence.
GEOS-Chem is the CTM which I am interested in and is a 3-D global CTM for atmospheric composition with transport driven by meteorological input from the Goddard Earth Observing System (GEOS) of the NASA Global Modeling and Assimilation Office (GMAO).
GEOS-Chem simulates more than 100 chemical species from the earth's surface up to the top of the atmosphere and can be used in combination with remote and in-situ sensing data to give a confident estimate of elements in remote areas.

Combining satellite and modelled data could allow for improved accuracy in records of natural Australian chemical sources as well as allow analysis of past and potential impacts of climate and environmental changes.
Due to the low availability of in-situ data covering most of the Australian outback an analysis of the modelled and remote sensed data could provide improved accuracy for global records of emissions which would in turn improve the accuracy or confidence of many models which are constrained or driven by these global inventories.

