\section{Future schedule}

\subsection{2016}

Running a GEOS-Chem simulation and storing available VOCs and comparing with the available in-situ measurements as well as satellite data over Australia will be the primary focus of my first six months in 2016.
The methods for comparison are mathematically complex in order to account for various reflectances and extinction effects as well as instrument sensitivities and understanding and utilising these concepts will take up the majority of this time.
Writing up the analysis and producing a paper will become the focus once the method of comparison is finalised.

Secondarily I will be working on improving the GEOS-Chem model to allow higher resolution simulations over Australia and New Zealand.
This may take longer than six months and will involve collaboration with several people involved in GEOS-Chem development (TODO:list?).

Finally I will be finalising ozone analysis from the in-situ ozonesonde measurements and looking at GEOS-Chem ozone profiles in the same areas, as well as finalising Australian dust analysis and looking for improvements made by running at the finer resolution.
