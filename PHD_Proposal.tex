\section{PHD Proposal}

\subsection{Improved Modelling resolution}
Higher resolution modelling over Australia will allow for both improved validation with in situ measurements, and regional and city scale simulation analysis. 
Dust simulation in particular can be greatly improved with finer surface wind resolution and will be one of the areas I will focus on with validation against both in situ and satellite measurements of AOD and dust concentrations across Australia.
One of my goals for my PHD will be to improve the GEOS-Chem model by setting up meteorological fields with higher horizontal resolution over Australia and NZ.
This has been done already over China, North America, and Europe (\cite{Chen_2009}, 2 more cites needed) but not for anywhere in the southern hemisphere.

\subsection{Ozone intrusions}
Ozone is present in the troposphere due to a variety of dynamical and photochemical processes, including downward  transport from the ozone-rich stratosphere and anthropogenic pollution.
Ozone-rich air mixes irreversibly down from the stratosphere during meteorologically conducive conditions \citep{Sprenger2003,Mihalikova2012}; these are referred to as Statosphere to Troposphere Transport events (STTs). 
In the extra-tropics, STTs most commonly occur during synoptic-scale tropopause folds \citep{Sprenger2003} and are characterised by tongues of high PV air descending to low altitudes.
These tongues become elongated and filaments separate from the tongue which mix into tropospheric air.
Stratospheric ozone brought deeper (lower) into the troposphere is more likely to affect the surface ozone budget and tropospheric chemistry \citep{Zanis2003}.

When looking at a vertical profile of ozone the effect of an STT is clear, with the right filter we should be able to quantitatively observe and characterise STTs.
Using a Fourier filter on the (TODO DESCRIBE DATA AND OWNERS) will give us the first southern hemispheric analysis of STTs.

\subsection{Isoprene and VOCs}
Another source(the main source) of ozone in the lower troposphere is chemical formation following emissions of precursor gases, including volatile organic compounds (VOCs) like isoprene, and nitrogen oxides (NOx).
Isoprene is commonly measured by proxy through satellite HCHO vertical column densities (VCDs) (cite: Isoprene emissions in Africa inferred from OMI observations).

\subsection{Dust emission and deposition validation}
Notes from Paper and validation of GOES dust simulation?
  
  
