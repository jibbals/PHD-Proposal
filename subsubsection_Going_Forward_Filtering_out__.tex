\subsubsection{Going Forward}

Filtering out the non biogenic sources of HCHO over Australia will be performed similarly to (TODO GET SOURCE) \cite{Marais_2012}. 

Atmospheric HCHO data exists over Wollongong from a Fourier transform spectrometer in operation since 1996, and some Australian measurements exist from campaigns such as the Measurements of Urban, Marine and Biogenic Air (MUMBA).
However these do not give a full overview of the continent's emissions. Satellite-based observations cover the entire continent, although with larger uncertainties involved due to horizontal resolution, cloud cover, and various other factors. 

Determination of purely biogenic emissions from satellite-based observations is possible by filtering out biomass burning using the fire counts and AOD.
The AATSR, Aqua, and Terra satellites have fire counts which can be used to determine when HCHO is caused by burnings or smoke plumes, as well as accounting for anthropogenic emissions (gas flaring).
These data can be used to filter satellite measurements in order to examine purely biogenic emissions and then determine biogenic isoprene emissions.
This method of inference has been used successfully in several countries including North America\cite{Palmer_2003}, South America \cite{Barkley_2013}, and Western Africa \cite{Marais_2012}.
Using the same methodology to retrieve estimates of isoprene emissions over Australia will be one of the focuses of this thesis.

Satellite HCHO measurements exist since 2004 (OMI) and 2006 (GOME2), and the ground based measurements over Wollongong exist since 1996. The ground based can be used as partial validation of models and satellite data. 
Combining the long record of measurements from Wollongong with the decadal-scale satellite data and chemical transport modelling in this thesis will allow the first large-scale quantification of isoprene emissions in Australia, paving the way for more accurate ozone projections.
