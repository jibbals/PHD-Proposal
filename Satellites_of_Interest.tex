\subsection{Satellites}

Several satellites provide long term trace gas data with near complete global coverage, including the ERS-2 launched in April 1995 which houses the GOME ultraviolet and visible (UV-Vis) spectrometer, the AURA launched in July 2004 which houses the OMI UV-Vis spectrometer, the MetOp-A and B launched in October 2006 and September 2012 respectively both housing a GOME-2 UV-Vis spectrometer.
These satellites are on Low Earth Orbit (LEO) trajectories and overpass any area up to once per day. 
They record near nadir reflected spectra between around 250-700~nm split into spectral components at around $0.3$~nm in order to calculate trace gases including O$_3$, NO$_2$, and HCHO.
An example of a spectrum retrieved from the GOME-2 instrument is given in figure \ref{fig:gomeproducts}.

Satellite retrievals of atmospheric HCHO are made either by fitting backscattered spectrum absorption (eg: \citet{Chance_2000}) or by using DOAS (eg: \cite{Leue_2001}).
This integral of HCHO abundance along the viewing path is called the slant column (SC), and the air mass factor (AMF) is the ratio of the SC to the vertical column.
The AMF is calculated by correcting the satellite viewing angle and the solar zenith angle for atmospheric scattering and surface albedo.

Horizontal transport 'smears' this signal so that source location would need to be calculated using windspeeds and loss rates.
In high NOx environments where HCHO has a lifetime on the order of 30 minutes HCHP can be used to map isoprene emissions with spatial resolution from 10-100 kms, which is comparable to satellite resolutions.
For conditions where VOCs have a lifetime of days determining the major HCHO contributors requires a complex inversion to map HCHO columns to VOC emissions.
This is due to the 'smearing' of source emissions at a greater range (order of 1000 kms) than the satellite column resolution.

Instruments including the MODIS on board the AQUA and TERRA satellites are able to determine aerosol optical depth (AOD), a measure of atmospheric scatter and absorbance. 
An AOD of under 0.05 indicates a clear sky, while values of 1 or greater indicate increasingly hazy conditions.
This is an important atmospheric property allowing us to track dust storms and pollution events as well as determine where measurements from other instruments may be compromised by high AOD.
Satellite measured AOD requires validation by more accurate ground based instruments like those of the Aerosols Robotic Network (AERONET) which uses more than 200 sun photometers scattered globally. 