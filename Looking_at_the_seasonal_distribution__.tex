
Looking at the seasonal distribution of dust emissions and deposition also shows agreement with the AERONET dust dominated sites.
While Australian summers are can be hot and dry, the majority of dust activity occurs in September and October, as can be seen in figure \ref{fig:seasonaldust}

A comparison between 45 seperate CoDii dust watch sites showed little correlation to modelled dust concentrations.
CoDii sites are single point measurements, looking at just the surface dust concentration, while the AERONET represents AOD over the whole column which is more directly comparable to the model.
Similarly to AERONET, satellite AOD represents column totals and may be a more apt source of model verification.

Several model inaccuracies occur due to the low resolution of the GEOS-Chem simulation.
A horizontal grid does not capturing the sub grid wind speed distribution \cite{Ridley_2013}, which leads to an underestimation of dust production.
Another problem lies in comparing area averages against in-situ data.
Using area averages limits the model's ability to capture small scale dust events and smears out the effect of large events.

The simulation from 2005 until 2012 recorded an average of almost 50 Tg yr$^{-1}$ of dust emitted from Australia, with about 37 Tg yr$^{-1}$ deposition, suggesting more than 10 Tg yr$^{-1}$ of dust is exported offshore.
Within the simulation 30 Tg yr$^{-1}$ of Australia's emitted dust is sourced from the Eyre basin.

Australian dust could play a large role in oceanic primary production through algal blooms in the southern ocean.
Using modelled estimates of deposition with and without the Eyre basin dust source allows a minimal bound estimate of Australia's direct contribution to the total southern oceanic dust deposition.
