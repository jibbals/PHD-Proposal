
Looking at the seasonal distribution of dust emissions and deposition also shows agreement with the AERONET dust dominated sites.
While Australian Summers are can be hot and dry, the majority of dust activity occurs in September and October, as can be seen in figure \ref{fig:seasonaldust}

A comparison between 45 seperate CoDii dust watch sites showed little correlation to modelled Dust AOD.
This could be due to a failure of the model resolution which may miss small dust events which could dominate dust AOD unless stations are situated close to the Eyre basin.

The simulation from 2005 until 2012 recorded an average of almost 50 Tg yr$^-1$ of dust emitted from Australia, with about 37 Tg yr$^-1$ deposition, suggesting more than 10 Tg yr$^-1$ of dust is exported offshore.
Within the simulation 30 Tg yr$^-1$ of Australia's emitted dust is sourced from the Eyre basin.
Austrlian dust could play a large role in oceanic primary production through algal blooms in the southern ocean.

Using the difference between simulations with and without the Eyre basin allows a minimal bound on the estimated Australian dust deposition over the near southern ocean.
(TODO:This number is in my poster somewhere) 