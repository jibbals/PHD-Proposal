One source of information which covers the entirety of Australia is remote sensing performed by instruments on satellites which overpass daily, recording reflected solar radiation (and emitted terrestrial radiation).
These can be used to quantify the abundance of several chemical species as well as estimate their vertical distribution over the land.
While satellite data is effective at covering huge areas, it only exists at a particular time of day, is subject to cloud cover, and generally does not have fine horizontal or vertical resolution.
Concentrations retrieved from satellite have large uncertainties.

The existence of satellite data covering remote areas provides an opportunity to develop more robust models of global climate and chemistry.
Natural emissions from areas with little anthropogenic influence and no ground based measurements characterise the majority of Australian land mass \cite{VanDerA_2008}.
Understanding of emissions from these areas is necessary to inform national policy on air pollution levels.
Satellite data allow us to verify large scale model estimates of natural emissions.
These measurements can be used to improve models, inform national policy, and predict harmful events.

This thesis will combine satellite and ground based atmospheric measurements with chemical transport modelling to clarify the impact of Australian natural emissions on atmospheric composition and chemistry.
In the following subsections the main atmospheric species to be analysed in this thesis are discussed.
Then the satellite and modelling techniques to be used will be examined.
