\subsection{Dust}

Australia is the greatest source of dust in the southern hemisphere producing around 120~Tg yr$^{-1}$ \cite{Li_2008}, however model validation and analysis over Australia is relatively scarce with more focus applied to the northern hemisphere \cite{Duncan_Fairlie_2007,Ridley_2013}.
Atmospheric dust has many direct effects including reduced surface insolation, mineral transfer to remote ocean regions, and health degradation in populated areas \cite{Shao_2007}.
Direct and indirect effects of dust have many implications which are not fully understood, with many models still struggling to explain the atmospheric cycling of dust at larger scales \cite{Rotstayn_2011}.

Australian dust emissions involve various weather conditions, convolving the ENSO cycle with flooding, droughts, and winds.
Rivers and rain build up the particulate matter in many areas, these are referred to as fluvial deposits.
Fluvial deposits in the Eyre basin increase the dust base load, which will only have mobility during suitable dry weather conditions.
These deposits are saltated and transported by strong winds\cite{Zender_2003}.
Saltation is the process of small particles being blown into larger particles and causing them to become airborne.

Synoptic scale measurements of dust concentrations in Australia are made by the Bureau of Meteorology (BOM) and can be used to estimate dust transport caused by large storms. 
These estimates exemplify the large variability in Australian annual dust transport with single storms moving up to 2.5 Tg of dust off shore in a single day.
Yearly dust emissions in Australia are somewhere between 10 and 110 Tg yr$^{-1}$.

Dust plays a large role in the oceanic carbon cycle, as dust is a major source of oceanic iron (Fe) deposition.
Some regions in the ocean are high in nutrients, but low in chlorophyll (HNLC), due to a lack of Fe.
Oceanic carbon cycling is a complex system in which Fe is a limiting factor, required by plankton in order to fix atmospheric nitrogen into a more bioavailable form such as ammonia.
Atmospheric deposition into the oceans is a very poorly constrained variable in global models \cite{Grand_2015}.
Model estimates of trace element oceanic deposition are required to quantify the atmospheric impact due to a dearth of in situ measurements in remote open ocean regions.

Measurements of dissolved iron (DFe) at very low concentrations like those found in surface ocean waters are very easily contaminated, which has contributed the the fragmentary and scarce nature of DFe ocean data sets \cite{Rijkenberg_2014}.
Recent analysis of the US Climate Variability and Predictability (CLIVAR)-CO$_{2}$ Repeat Hydrography Program predicted total deposition flux with uncertainty at a factor of 3.5 \cite{Grand_2015}.
Some headway has been made with the recent GEOTRACES program which has several transects of the major oceans and measures trace elements over multiple depths including Al, Ba, Cu, Cd, Fe, Mn, Ni, Pb, and Zn.
  
Total iron (TFe) emissions from dust and combustion sources are estimated (by average of several global models) at approximately 35~Tg yr$^{-1}$ and 2 Tg yr$^{-1}$ respectively. A two fold increase in Fe dissolution may have occurred since 1850 due to increased anthropogenic emissions and atmospheric acidity.
This increase may revert by 2100 due to the affects of emission regulations \cite{Myriokefalitakis_2015}.
Dust, TFe and DFe have strong temporally and spatial variability, with changes having most impact upon HnLC regions.

Another environmental impact of dust is its contribution to fine particulate matter in the atmosphere.
Several studies have shown that long term exposure to fine particulate matter (PM2.5) increases mortality. 
Estimates of yearly premature deaths related to PM2.5 are $\sim$ 2-3 million \cite{Hoek_2013, 19627030, Silva_2013, Lelieveld_2015}.   
These estimates are made using global atmospheric models or model ensembles to quantify population exposure before applying epidemiological models to estimate the increased death rates.
The main source of uncertainty in premature death rates arises from the difference and uncertainties between and within the atmospheric models.

Dust affects global climate change through direct radiative forcing.
Uncertainties in the atmospheric dust concentrations make accurate determination of radiative forcing from other sources more difficult \cite{IPCC_2013_chap8}.