\subsection{Isoprene and formaldehyde (HCHO)}

The primary VOCs are methane and isoprene, each comprising around a third of the global total.
However, methane is very long lived (years) and is well mixed in the atmosphere while isoprene levels are very volatile and spatially diverse. Estimates put global isoprene emission at roughly 550 Tg yr$^{−1}$ [Guenther et al., 2006, Monks et al., 2014], emitted mostly by trees and shrubs during the day.
Isoprene is hard to directly measure due to its short lifetime, instead formaldehyde (HCHO) is often used as a proxy \cite{Marais_2012,bauwens2013satellite}.

Biogenic isoprene emissions are far greater than anthropogenic emissions of VOCs \cite{Guenther_2006}. 
The lack of accuracy in BVOC emissions estimates has a large effect on determining with confidence the sources and distribution of pollutants including ozone and organic aerosols.
Accuracy in VOC measurements is important: it has been shown that even the diurnal pattern of isoprene emissions has an effect on modelling ground level ozone \cite{Hewitt_2011,Fan_2004}.
These uncertainties could explain why models of HCHO over Australia are poor at reproducing satellite measurements \cite{Stavrakou_2008}.

A widely used estimate of global isoprene emissions is the Model of Emissions of Gases and Aerosols from Nature (MEGAN), which uses leaf area index, forest canopy modelling, and plant functional type emission factors to generate terrestrial isoprene emissions.
Emission factors represent quantities of a pollutant released to the atmosphere through an associated activity.
For example, an emission factor for isoprene within a forest would include the requirement of sunshine and suitable temperature.
\citet{Guenther_2006} shows how the Australian outback is among the worlds strongest isoprene sources with large forests near coastal cities having emission factors greater than 16 mg m$^{-2}$ h$^{-1}$ (see figure \ref{fig:meganisoprene}).

Using satellite data from 1995(see section \ref{satellites}) we have a broad measure of seasonal and interannual variability of HCHO over Australia.
These records can be compared with modeled estimates of HCHO which is used as a proxy to estimate isoprene emissions.
This has been done in North America, South America, and Africa, with satellite and aircraft data combined for validation \cite{Millet_2006, Marais_2014}.

The methodology for calculating VOCs from HCHO is laid out in Palmer et al. (2003), and takes into account the expected lifetime and reaction rates of the precursor VOCs and HCHO.
Assuming HCHO is produced quickly from short-lived intermediates we get
\begin{eqnarray*}
VOC_i \overset{k_i}{\rightarrow} Y_i HCHO
\end{eqnarray*}
Where $Y_i$ is HCHO yield per C atom.
Then assuming a steady state of atmospheric HCHO ($\Omega$ molecules $cm^{-2}$) produced by oxidation of VOCs (VOC$_i$) and no horizontal transport:
\begin{eqnarray*}
\Omega = \frac{1}{k_{HCHO}} \sum_{i} Y_i E_i
\end{eqnarray*}
Where i indexes a chemical species, and $E_i$ is emission fluxes ( C atoms $cm^{-2}s^{-1}$).

Inferring the VOC emissions then requires estimates of the HCHO yield (Y$_i$) which can be attained via modelling as layed out in \citet{Millet_2006}.

During low NO$_X$ environments, the precursor HCHO has a longer lifetime (days).
This allows horizontal transport to occur and complicates the algorithms as an inversion is required in order to work out the HCHO source.
