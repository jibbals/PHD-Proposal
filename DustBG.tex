\subsection{Dust}

\begin{quote}
Aerosols partially offset the forcing of the WMGHGs and dominate the uncertainty associated with the total anthropogenic driving of climate change \cite{IPCC_2013_chap8}.
\end{quote}

Australia is the largest source of dust in the southern hemisphere producing around 120~Tg$a^{-1}$ \cite{Li_2008}, however model validation and analysis over Australia is relatively scarce with more focus applied to the northern hemisphere.
Aeolian dust has many direct effects including reduced surface insolation, mineral transfer to remote ocean regions, and health degradation in populated areas.
Direct and indirect effects of dust have many implications which are not fully understood, with many models still struggling to explain the role of dust at larger scales \cite{Rotstayn_2011}.

Synoptic scale measurements of dust plumes in Australia are kept by the BoM and can be used to estimate dust transport caused by large storms(\cite{Leys_2011}, todo more cites here). 
These estimates exemplify the large variability in Australian annual dust transport with single storms moving up to 2.5 Tg of dust off shore in a single day, with impacts on population health, oceanic carbon production, and erosion cycles \cite{Leys_2011,Shao_2007}.

The Australian dust story is one of extremes, convolving the ENSO cycle with flooding, droughts, and winds.
Fluvial deposits in the Eyre basin are required to provide a dust base, which will only have mobility during suitable dry weather conditions, which are saltated and transported by strong winds\cite{Zender_2003}.

Oceanic carbon cycling is a complex system in which iron (Fe) is a limiting factor, required by diazotrophs in order to fix atmospheric Nitrogen into a more bioavailable form such as ammonia.
Atmospheric deposition into the oceans is a very poorly constrained variable in global models \cite{Grand_2015}.
Various estimates of trace element oceanic depositions are required due to a dearth of in situ measurements in remote open ocean regions.

Measurements of dissolved iron (DFe) at very low concentrations like those found in surface ocean waters are very easily contaminated, which has contributed the the fragmentary and scarce nature of DFe ocean data sets \cite{Rijkenberg_2014}.
Recent analysis of the US Climate Variability and Predictability (CLIVAR)-CO$_{2}$ Repeat Hydrography Program predicted total deposition flux with uncertainty at a factor of 3.5 due mostly to assumed Aluminium solubility and residence time \cite{Grand_2015}.
Some headway has been made with the recent GEOTRACES program which has several transects of the major oceans and measures trace elements over multiple depths including Al, Ba, Cu, Cd, Fe, Mn, Ni, Pb, and Zn.
  
Total iron (TFe) emissions from dust and combustion sources are estimated (by average of several global models) at approximately 35~Tg yr$^{-1}$ and 2Tg yr$^{-1}$ respectively. A three fold increase in Fe dissolution may have occurred since 1850 due to increased anthropogenic emissions and atmospheric acidity, with a 30\% decrease projected from current levels by 2100 \cite{Myriokefalitakis_2015}.
Dust, TFe and DFe have strong temporally and spatial variability, with changes having most impact upon high nutrient low chlorophyll (HNLC) regions.

While the affects of Australian dust on primary production in the southern oceans remains to be clarified, another issue is the problem of fine particulate matter in the air.
Several studies show that long term exposure to fine particulate matter (PM2.5) increases mortality. 
Estimates put yearly premature deaths related to PM2.5 at $\sim$ 2-3 million \cite{Hoek_2013, 19627030, Silva_2013, Lelieveld_2015}.   
These estimates are made using global atmospheric models (in the case of \citet{Silva_2013} an ensemble of 14 models from the Atmospheric Chemistry and Climate Model Intercomparison Project (ACCMIP)) in order to work out population exposure before applying epidemiological models to estimate the increased death rates.
To work out confidence intervals and ensure statistical significance Monte Carlo methods are used within the simulations, however the main source of uncertainty arises from the difference and uncertainties between and within the atmospheric models.

While dust in Australia is currently episodic, a better understanding of the environment could help predict when dust storms are likely to impact populated areas.

(TODO: How to link to Satellites section? move sections?)