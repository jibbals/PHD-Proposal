\section{GEOS-Chem}
The GEOS-Chem model is a global model of atmospheric composition using meteorological observations from the Goddard Earth Observing System (GEOS) of the NASA Global Modeling Assimilation Office (GMAO). It is developed by Harvard University staff as well as users and researchers. 

The Current version of GEOS-Chem (v10-01) runs globally with 47 vertical levels from the surface up to the top of the atmosphere (TOA). 
Until recently the model could only be run at several specified horizontal resolutions defined at compile time, however this requirement is removed and the model can be run at any resolution \cite{Long_2015}.
Datasets are assimilated as boundary conditions and drivers of this CTM are of various resolutions and exist over several areas at high resolution (0.25$^{\circ}$ x 0.3125$^{\circ}$ over North America, China, and Europe).

\subsection{Improved Modelling}
One of my goals for the PHD will be to improve the GEOS-Chem model by setting up meteorological fields with higher horizontal resolution over Australia and NZ.
Higher resolution modelling over Australia will allow for improved validation with in situ measurements (cite chen nested-grid paper). 
Dust simulation can be greatly improved with finer surface wind resolution and will be one of the areas I will focus on with validation against both in situ and satellite measurements of AOD and dust concentrations.

