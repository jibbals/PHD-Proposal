\subsection{Improved GEOS-Chem Modelling resolution}
\subsubsection{Preliminary Work}
A preliminary understanding of the mechanics required to both build and run the model at various resolutions has been achieved through active modelling and work done at the seventh international GEOS-Chem conference.
This will help with initial project planning.

\subsubsection{Going Forward}
Higher resolution modelling over Australia will allow for both improved validation with in situ measurements, and regional scale simulation analysis. 
Dust simulation in particular can be greatly improved with finer surface wind resolution and will be one of the areas I will focus on with validation against both in situ and satellite measurements of AOD and dust concentrations across Australia.
During this PHD the GEOS-Chem model will be improved by setting up meteorological fields with higher horizontal resolution over Australia and NZ.
This has been done already over China \cite{Chen_2009,Wang_2004}, North America \cite{Zhang_2012}, Europe, and the Amazon (TODO: Europe cite?? needed).

First sufficient disk space will be required to hold several years worth of high resolution meteorological data and emissions, this is expected to be to the order of 100s of gigabytes.
After determining the desired spatial and temporal boundaries for improved resolution, raw data will be downloaded, verified, and re-gridded using partially written code from Bob Yantosca.
All of the data will need be compared to other emissions estimates in order to ensure that we have an acceptable inventory for Australian emissions.
Finally, emissions data needs to be reworked over the domain, ensuring emissions factors and chemistry is maintained.
This last requirement may be eased by the new HEMCO module implemented within the ESMF environment (compatible with GEOS-Chem v10) which recalculates emissions on any user specified grid.
  