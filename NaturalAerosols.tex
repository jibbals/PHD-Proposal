\subsection{Natural gas and aerosol emissions in Australia}
Natural emissions in Australia are important due to the large population sparse areas from where dust, isoprene, and other important biogenics are released.
These trace gasses and aerosols affect surface ozone levels and can alter the radiative and particulate matter distribution of the atmosphere with harmful results.
Ozone in the lower atmosphere is a serious hazard that causes health problems \cite{Hsieh_2013}, damages agricultural crops worth billions of dollar \cite{Avnery_2011}, and increases the rate of climate warming \cite{IPCC_2013_chap8}. 

The Australian outback is described by extremely large and diverse environments. 
Uncertainty in estimates of atmospheric gas and particulate densities and distributions stem from a lack of in-situ measurements.
\citet{Guenther_2006} shows how the Australian outback is among the worlds strongest isoprene sources with large forests near coastal cities emitting greater than 16 mg m$^{-2}$ h$^{-1}$ (see figure \ref{fig:meganisoprene}).

  