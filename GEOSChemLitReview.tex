\subsection{GEOS-Chem}

The GEOS-Chem model is a global model of atmospheric composition using meteorological observations from the Goddard Earth Observing System (GEOS) of the NASA Global Modeling Assimilation Office (GMAO). 
It is developed by Harvard University staff as well as users and researchers. 
Several driving meteorological fields exist with different resolutions, following is a description of these fields and their availability. 
GEOS-4 natively has 1 by 1.25$^\circ$ horizontal resolution with 55 vertical levels and is available from 1985 to 2006. 
MERRA natively has 0.5 by 2/3$^\circ$ horizontal resolution with 72 vertical levels and is available between 1979 to 2010.
GEOS-5 natively has 0.5 by 2/3$^\circ$ horizontal resolution with 72 vertical levels and is available from Dec. 2003 until mid 2013.
GEOS-FP natively has 0.25 by 0.3125$^\circ$ horizontal resolution with 72 vertical levels and is available from 2012 onwards.
MERRA and GEOS-FP have temporal resolution of 3 hours, except for surface quantities and mixing depths which are resolved to 1 hour.
GEOS-4 and GEOS-5 have temporal resolution of 6 hours, except for surface quantities and mixing depths which are resolved to 3 hours.

Since the release of v10-01, GEOS-Chem is grid-independent allowing any horizontal resolution to be run globally. 
It is also compatible with Earth System Modeling Framework (ESMF), allowing GEOS-Chem to run as the chemical module for Earth System Models (ESMs). 
GEOS-Chem allows distributed-memory parallelisation which can improve chemistry simulation performance almost linearly with increased cores (at least up to 240) \cite{Long_2015}.

GEOS-Chem can be run in a nested fashion, with high resolution in regional areas (0.25$^{\circ}$ x 0.3125$^{\circ}$ over North America, or Asia, or  0.5$^{\circ}$ x 0.667$^{\circ}$ over North America, Asia, or Europe) while a lower global resolution is implemented allowing more reasonable run times.
  
  
  