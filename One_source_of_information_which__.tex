One source of information which covers the entirety of Australia is remote sensing performed from satellites which overpass daily and record reflected solar spectra.
These can be used to garner details of several chemicals as well as estimates of their distribution in vertical columns over the land.
While satellite data is very good for covering huge areas (the entire earth) it only exists at a particular time of day, is subject to cloud cover, and generally does not have fine horizontal or vertical resolution.
Additionally vertical profiles from satellites are estimates based on forward radiative transfer modelling which is hampered by the sparse in-situ data.
This leaves us with a best guess for chemical concentrations at any altitude which may need verification or adjustment.

Satellites can use Differential Optical Absorption Spectroscop (DOAS) techniques with radiative transfer calculations on solar radiation absorption spectra to measure trace gases in vertical columns.
DOAS methods can be heavily influenced by the initial estimates of a trace gas profile (the a priori) which is often produced by modelling, so when comparing models of these trace gases to satellite measurements extra care needs to be taken to avoid introducing bias from unrealistic a priori assumptions.
A way to remove these a priori influences in order to compare models and satellites is through the satellite's averaging kernal, which is a measure of the sensitivity of the instrument to the trace gas's radiance at various heights multiplied by the sensitivity of the DOAS technique's forward radiative transfer model (RTM) to the amount of trace gas at various heights near the a priori \cite{Eskes_2003}.
The RTM used in DOAS techniques is based on Beer's law relating the attenuation of light to the properties of the medium it travels through.
$$ I = I_0 \exp \left( \Sigma_i \int \rho_i \beta_i ds \right) $$
Where I is radiation intensity, i represents a chemical species index, $\rho$ is a species density, and $\beta$ is the scattering and absorption cross section area.
The forward RTM actually used by satellites involves functions representing extinction from Mie and Rayleigh scattering, and the efficiency of these on intensities from the trace gas under inspection, as well as accounting for various atmospheric parameters which may or may not be estimated (e.g. albedo).
Finally to convert the trace gas profile from a reflected solar radiance column (slanted) into a purely vertical column requires calculations of an air mass factor (AMF) due to the non strict path of light measured by the instrument.
The AMF is normally a scalar value for each horizontal grid point which will equal the ratio of the total vertical column density to the total slant column density.

The existence of satellite data covering remote areas provides researchers an opportunity to analyse truly global information and put together a better idea of global climate and chemistry.
Comparing global models of atmospheric chemistry with the satellite data to test our understanding of large scale natural chemistry and potentially improve those models and predict harmful events is a worthwhile goal.