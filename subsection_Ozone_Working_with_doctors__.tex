\subsection{Ozone}
Working with doctors Simon Alexander and Robyn Schofield, records of ozone profiles over Davis, Macquarie, and Melbourne have been analysed and filtered to quantitatively determine instances of Stratosphere to Troposphere Transport events (STTs).

The vertical profiles of ozone volume mixing ratio are linearly interpolated to a regular grid with 20m resolution up to 14km altitude and are then bandpass filtered so as to retain perturbations on altitude scales between 0.5km - 5km. The choice of band limits is set empirically, but we note that to define an STT event, a clear increase above the background ozone level is needed, and a vertical limit of $\sim 5$~km removes seasonal-scale effects. The ozone perturbation profile is analysed at altitudes from 2~km above the surface (to avoid surface pollution events) and 1~km below the tropopause (to avoid the sharp transition to stratospheric air producing spurious false positives). Perturbations above the 99~th percentile (locally) of all ozone levels are initially classified as STT events.

TODO: More Outcomes and Results.