One source of information which covers the entirety of Australia is remote sensing performed from satellites which overpass daily and record reflected solar spectra.
These can be used to garner details of several chemicals as well as estimates of their distribution in vertical columns over the land.
While satellite data is very good for covering huge areas (the entire earth) it only exists at a particular time of day, is subject to cloud cover, and generally does not have fine horizontal or vertical resolution.
Additionally vertical profiles from satellites are estimates based on forward radiative transfer modelling which is hampered by the sparse in-situ data.
This leaves us with a best guess for chemical concentrations at any altitude which may need verification or adjustment.

Satellites can use Differential Optical Absorption Spectroscop (DOAS) techniques with radiative transfer calculations on solar radiation absorption spectra to measure trace gases in vertical columns.
DOAS is based on Beer's law relating the attenuation of light to the properties of the medium it travels through. 
$$ I = I_0 \exp \left( \Sigma_i \int \rho_i \beta_i ds \right) $$
Where I is radiation intensity, i represents a chemical species index, $\rho$ is a species density, and $\beta$ is the scattering and absorption cross section area.
The formula actually used by satellites involves further functions representing scattering efficiency and extinction by Mie and Rayleigh scattering.
Calculations of slant columns and 'air mass factors' is required to determine the effect of the non strict path of light measured by the instrument.

The existance of satellite data covering remote areas provides researchers an oportunity to analyse truly global information and put together a better idea of global climate and chemistry.
Comparing global models of atmospheric chemistry with the satellite data to test our understanding of large scale natural chemistry and potentially improve those models and predict harmful events is a worthwhile goal.