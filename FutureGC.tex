\section{Future schedule}

\subsection{2016}

Initially sorting the HCHO satellite data into manageable formats and determining an appropriate filtering scheme for Australia will take place.
Running the simulation and then comparing satellite and model HCHO will follow.

The methods for comparison are mathematically complex in order to account for various reflectances and extinction effects as well as instrument sensitivities and understanding and utilising these concepts will take some time.

Following these tasks, the inversion of HCHO will take place in order to estimate isoprene emissions.
The results from this will contribute to planning for a field campaign to take place in south eastern Australia in 2017 focused on biogenics as well as a possible aircraft campaign in 2018 led by the National Center for Atmospheric Research (NCAR).

Finalising ozone analysis from the in-situ ozonesonde measurements and submitting this as a paper to the journal of Atmospheric Chemistry and Physics (ACP) will also take place early in 2016

\subsection{2017}
Setting up a nested grid within the GEOS-Chem model will allow higher resolution simulations over Australia.
The new resolution will need to be tested and compared against benchmark simulations in order to verify that the implementation has not caused any undesired effects.
This will involve collaboration with several people involved in GEOS-Chem development and the National Environmental Science Programme (NESP) Clean Air and Urban Landscapes Hub.

Thesis writing will be an ongoing task throughout the course of this PhD.