\subsection{Tropospheric ozone}

Ozone is one of the toxic trace gases which increase mortality rates when populations are exposed for extended periods of time.
The amount of premature deaths per year due to atmospheric ozone exposure has recently been estimated at $\sim$150-470 thousand \cite{Silva_2013, Lelieveld_2015}.
Long term effects of ozone overexposure increase the risk of respiratory disease and may also increase other cardiopulmonary risks \cite{Jerrett_2009}.

Ozone and is one of the six substances monitored by the Ambient Air Quality (AAQ) National Environment Protection Measure (NEPM), which is the Australian framework for air quality measurement and reporting aiming for ``adequate protection of human health and well-being.''
NEPM AAQ benchmarks are shown in figure \ref{fig:nepm}.

Photochemical oxidation of CO, CH$_4$, and other VOCs while NOx (NOx $=$ NO, NO$_2$) is present, as well as downward stratospheric transport, are the two main sources of tropospheric ozone \cite{Stevenson_2006}.
HCHO and NO$_2$ regulate tropospheric oxidation capacity through O$_3$ production, as well as being health hazards.
The HCHO/NO$_2$ ratio can be used to determine whether surface O$_3$ is NO$_2$ or VOC limited \cite{Mahajan_2015}.
NO$_2$ is a common pollutant in populated areas, released primarily by combustion in power generation and transport. 
In Australia VOCs and NOx are largely emitted from biogenic sources, although some anthropogenic influence is seen from biomass burnings in the Northern Territory \cite{Guenther_2006,van_der_A_2008}.

Photolysis of NO$_2$ forms NO + O which reacts with O$_2$ to form O$_3$ which reacts with NO to form NO$_2$ + O$_2$, these reactions reach an equilibrium where O$_3$ is proportional to the ratio between NO$_2$ and NO.
VOCs affect the NO$_2$ to NO ratio which changes the equilibrium and can increase O$_3$ creation. 
One dominant precursor to ozone is isoprene (C5H8), which in the atmosphere reacts rapidly with hydroxyl radicals (OH) to form peroxy radicals (O$_2$), which react with nitrogen oxides and can lead to ground-level ozone formation similarly to the CO reaction listed prior.
The following formulae show an example of this with CO however similar reactions occur for many VOCs:
\begin{eqnarray*}
NO_2 + hv &\overset{k_1}{\rightarrow}& NO + O(3P) \\
O(3P) + O_2 &\overset{M}{\rightarrow}& O_3 \\
NO + O_3 &\overset{k_2}{\rightarrow}& NO_2 + O_2 \\
\left[O_3\right] &\sim& \frac{k_1}{k_2} \frac{\left[NO_2\right]}{\left[NO\right]} \\
OH + CO &{\rightarrow}& HOCO \\
HOCO + O_2 &{\rightarrow}& HO_2 + CO_2 \\
HO_2 + NO &{\rightarrow}& OH + NO_2 \\
\end{eqnarray*}
The balance of these reactions is:
\begin{eqnarray*} CO + 2O_2 + hv {\rightarrow} CO_2 + O_3 \end{eqnarray*}
Note that hv represents light being present within the system.

While photochemical production is the dominant source, stratosphere to troposphere transport (STT) of ozone is also important and climate change may drastically increase this quantity \cite{Hegglin_2009}.
The amount of ozone from STTs and photochemical production are estimated to be around 550 and 5100 Tg yr$^{-1}$, respectively \cite{Stevenson_2006}. 
The main ozone removal processes are chemical destruction and dry deposition.

Ozone-rich air mixes irreversibly down from the stratosphere during certain meteorological conditions \citep{Sprenger2003,Mihalikova2012}.
In the extra-tropics, STTs most commonly occur during synoptic-scale tropopause folds \citep{Sprenger2003} and are characterised by tongues of high Potential Vorticity (PV) air descending to low altitudes.
PV is a metric which can be used to determine whether a parcel of air is stratospheric based on it's local rotation and stratification.
These tongues become elongated and filaments separate from the tongue which mix into tropospheric air.
Stratospheric ozone brought deeper (lower) into the troposphere is more likely to affect the surface ozone budget and tropospheric chemistry \citep{Zanis2003,Langford_2009}.

  