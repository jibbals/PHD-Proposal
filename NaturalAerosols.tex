\subsection{Natural gas and aerosol emissions in Australia}

The Australian landscape is largely uninfluenced by human activity.
These environments are sources of naturally released trace gases which make up less than 1\% of earth's atmosphere.
Naturally occurring trace gases in the atmosphere can have a large impact on living conditions.
They react in complex ways with other elements (anthropogenic and natural), as well as affecting various ecosystems upon which life depends.
Natural emissions can drift over populous areas and influence local pollution levels in various ways.
Ozone can be produced when some natural trace gases interact with pollutants from petrol combustion, and in the lower atmosphere ozone is a serious hazard that not only causes health problems \cite{Hsieh_2013}, and billions of dollars worth of damage to agricultural crops \cite{Avnery_2011}, but also increases the rate of climate warming \cite{IPCC_2013_chap8}.
Particulate matter in the atmosphere is also a major problem, causing an estimated 2-3 million deaths annually \cite{Hoek_2013, 19627030, Silva_2013, Lelieveld_2015}.

Natural emissions can also alter the radiative and particulate matter distribution of the atmosphere, complicating simulations and increasing uncertainty when not properly accounted for.
The Australian outback includes large and diverse environments, which can have unique chemical sources.
Much of the landscape outside of urban areas is undeveloped and sparsely inhabited.
In Australia most long term air quality measurements are performed in or near large cities.
However, estimates of atmospheric gas and particulate densities and distributions over much of the rest of the continent are uncertain and lack in-situ measurements.
%For instance the Total Carbon Column Observing Network (TCCON) has sites at Darwin and Wollongong, and the Aerosols Robotic Network (AERONET) 