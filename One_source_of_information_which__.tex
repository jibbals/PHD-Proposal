One source of information which covers the entirety of Australia is remote sensing performed by instruments attached to satellites which overpass daily recording reflected solar spectra.
These can be used to garner details of several chemicals as well as estimates of their distribution in vertical columns over the land.
While satellite data is effective at covering huge areas (the entire earth) it only exists at a particular time of day, is subject to cloud cover, and generally does not have fine horizontal or vertical resolution.
Additionally, vertical profiles from satellites are estimates based on forward radiative transfer modelling which requires good in-situ data to reduce inaccuracy.
This results in a best guess for chemical concentrations at any altitude which may need verification or adjustment.

The existence of satellite data covering remote areas provides researchers an opportunity to analyse truly global information and develop more robust models of global climate and chemistry.
Natural chemical emissions from areas with little anthropogenic influence and no ground based measurements form the majority of Australian land mass \cite{VanDerA_2008}.
Detailed analysis of these areas is required to inform national policy on air pollution levels (TODO:Cite nepm? or remove todo).
Comparing global models of atmospheric chemistry with satellite data is a worthwhile goal allowing validation of large scale natural chemistry and model improvement, informed policy, and harmful event prediction.

In this thesis I will combine satellite and ground based atmospheric measurements with modelled emissions estimates to clarify the impact of Australian biogenic gases and aerosols on the population's health and atmosphere.