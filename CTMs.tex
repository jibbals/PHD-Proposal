\subsection{Chemical Transport Models}
Chemical Transport Models (CTMs) simulate production, loss, and transport of chemical species.
This is generally calculated using one or both of the Eulerian (box) or Lagrangian (puff) frames of reference.
CTMs normally solve the continuity equation while running chemical production and loss models for chemicals under inspection. 
The continuity equations describe transport of a conserved quantity such as mass, which, solved together with production and loss of a chemical forms the basis for a CTM.
This basis enables a record the chemical densities and transport over time as a model runs.
The continuity equation links a quantity of a substance to the field in which it flows and can be described by the formula:
\begin{eqnarray*}
    \frac{\partial \rho}{\partial t} + \nabla \cdot j &=& \sigma 
\end{eqnarray*}
Where $\rho$ is density, t is time, $\div$ is divergence, j is the flux of q (amount of q per unit area per unit time), and $\sigma$ is the generation of q per unit volume per unit time.
Note that $\sigma$ can be positive or negative due to sources and sinks.

The type of model best suited to modelling the entire earth uses the Eulerian frame of reference, where the atmosphere is broken up into 3-D boxes with densities and transports calculated and stored for arbitrary sequential steps in time at each location.
The mass balance equation must be satisfied in any realistic long term box model and is as follows: 
\begin{eqnarray*}
    \frac{dm}{dt} &=& \sum{sources}-\sum{sinks} \\
    &=& F_{in} + E + P - F_{out} - L - D 
\end{eqnarray*}
where m is mass of a chemical, E and D are emission and deposition, P and L are production and loss, and F is chemical transport in and out, as shown in figure \ref{fig:boxModel}.
Many chemical species interact with each other through chemical gradients, catalyst reactions, intermediary and direct production and loss. 
Any large chemical model will be solving this mass balance equation over highly coupled arrays of partial differential equations and this can be complicated and take a long time.

If measurements exist they can be fed into CTMs as boundary and/or initial conditions in order to examine and reproduce real world concentrations.
When initial measurements are uncertain then the simulated outcomes will also have uncertainty.
Chemical production and loss models for various species can be run while using meteorological data to drive the transport within these models.

GEOS-Chem is a well supported CTM with state of the science chemistry modelling which runs globally using a 3-D Eulerian frame of reference focused on atmospheric composition with transport driven by meteorological input from the Goddard Earth Observing System (GEOS) of the NASA Global Modeling and Assimilation Office (GMAO).
GEOS-Chem simulates more than 100 chemical species from the earth's surface up to the edge of space (0.01 hPa) and can be used in combination with remote and in-situ sensing data to give a verifiable estimate of atmospheric gases and aerosols.
It is developed by Harvard University staff as well as users and researchers. 
Several driving meteorological fields exist with different resolutions, the finest at 0.5 by 0.67$^\circ$ horizontally at hourly time steps with 72 vertical levels.


Combining satellite data with model outcomes allows us to test our understanding of natural processes and gives us a sanity check on what is going now and into the future over Australia.
Due to the low availability of in-situ data covering most of the Australian outback an analysis of the modelled and remote sensed data could provide improved accuracy for global records of emissions which would in turn improve the accuracy or confidence of many models which are constrained or driven by these global inventories.
