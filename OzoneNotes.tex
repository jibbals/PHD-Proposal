\section{Isoprene, Ozone, and other Biogenics}

\subsection{\cite{Zhang_2011}}
Improved estimate of the policy-relevant background ozone in the United States using the GEOS-Chem global model with 1/2 x 2/3 horizontal resolution over North America.

Policy-relevant background(PRB) ozone is defined as surface ozone in the absence of local anthropogenic emissions.
National Ambient Air Quality Standard (NAAQS) at this point has the standard of a maximum  8 hour average ozone of 75 ppbv, and is considering decreasing it to 60-70 ppbv, which approaches the PRB. Ozone has a lifetime of a few days in the continental boundary layer, but weeks in the free troposphere.

Model is run over three years(2006-2008), GEOS-Chem version 8-02-03, driven by GEOS-5 met fields ad .5 by .666 resolution. Biomass burning emissions from Global Fire Emission Database version 2(GFED-v2), lightning source imposed to be 6 Tg N / a.
Simulation is run with and without North American anthropogenic emissions, and another run without global anthropogenic emissions.

\subsection{\cite{Zhang_2014}}
Sources contributing to background surface ozone in the US
Intermountain West.

\begin{quote}
Models may overestimate ozone production in
fresh fire plumes because of inadequate chemistry and grid-
scale resolution.
\end{quote}

High ozone concentrations can be associated with stratospheric intrusions.
Especially in high altitudes and large scale subsidence like Sierra Nevada/Cascades and the Rocky Mountains.

NAAQS defined background ozone surface ppbv may be nearing the limit of actual background ozone which can be acheived through improvements in emissions located in the USA, Mexico, and Canada.

This background sans North American emissions can only be estimated through models, this paper looks at GEOS-Chem and CAMx regional model with GEOS-Chem boundary conditions.
Nested 1/2 by 2/3$^{\circ}$ with global 2 by 2.5$^{\circ}$ resolution on GEOS-Chem version 8-02-03 used for north America from 2006 to 2009.

Model results are compared to the ensemble of ozone observations at CASTNet sites in the western USA.

\begin{quote}
Stratospheric ozone is simulated with the Linoz linearized parameterization(McLinden et al., 2000) above the tropopause diagnosed by the GEOS-5 data and transported to the troposphere with the model winds. The resulting global cross-tropopause ozone flux is 490 Tg ozone $a^{−1}$, consistent with the range of 475$\pm$120 Tg $a^{−1}$ constrained by observations(McLinden et al., 2000).
\end{quote}

Using beryllium-7 observations, a cosmogenic tracer produced in the UTLS, it's shown that GEOS-Chem successfully simulates the $^7$Be observations, supporting the simulation of vertical transport in GEOS-Chem.

  
\subsection{NO2 and HCHO over India \cite{Mahajan_2015}}
NO2 and HCHO regulate tropospheric oxidation capacity through O3 production. 

if HCHO/NO2 ratio is less than 1 then Ozone is NO2 limited, if ratio is greater than 2 O3 is VOC limited.
Over most of India ratio is greater than 2 and NOx is increasing proportionally faster than HCHO.

Four satellites examined over 1996 to 2013, GOME, SCIAMACHY, OMI, GOME-2.
Average trend of HCHO over india is $1.51 \pm 0.44\%$ per year ($10.34 \pm 2.87 *10^{13} molecs/cm^2/yr$), and for NO2 is $2.2 \pm 0.73\%$ per year ($3.94 \pm 1.15 *10^{13} molecs/cm^2/yr$).
  
  
\subsection{Isoprene source inversion using HCHO \cite{bauwens2013satellite}}
Isoprene is the main biogenic hydrocarbon emitted with global flux of 400-600 Tg per year. Emissions are still largely uncertain due to scarce field measurements and spatiotemporal variability.

GOME2 HCHO columns for 07-12 compared against IMAGESv2 global CTM, inversion used to minimise difference and infer improved emissions inventory.
Australian HCHO emissions are suggested to be lowered by 19\%.
  