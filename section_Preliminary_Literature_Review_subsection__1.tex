\subsubsection{Oceanic Deposition}
Oceanic carbon cycling is a complex system in which iron is a limiting factor, required by diazotrophs in order to fix atmospheric Nitrogen into a more bioavailable form such as ammonia.
Atmospheric deposition into the oceans is a very poorly constrained variable in global models (\cite{Grand_2015} more CITES).
Various estimates of trace element oceanic depositions are required due to a dearth of in situ measurements in remote open ocean regions.

Measurements of dissolved iron (DFe) at very low concentrations like those found in surface ocean waters are very easily contaminated, which has contributed the the fragmentary and scarce nature of DFe ocean data sets \cite{Rijkenberg_2014}.
Recent analysis of the US Climate Variability and Predictability (CLIVAR)-CO$_{2}$ Repeat Hydrography Program predicted total deposition flux with uncertainty at a factor of 3.5 due mostly to assumed Aluminium solubility and residence time \cite{Grand_2015}.
Some headway has been made with the recent GEOTRACES program which has several transects of the major oceans and measures trace elements over multiple depths including Al, Ba, Cu, Cd, Fe, Mn, Ni, Pb, and Zn.