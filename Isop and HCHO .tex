\subsection{Isoprene and formaldehyde(HCHO)}

Biogenic isoprene emissions are far greater than anthropogenic emissions of VOCs \cite{Guenther_2006}. 
The estimates for isoprene emissions lack verification, especially over Australia where very little work has been done to measure biogenic VOCs (BVOCs) like isoprene.
This lack of accuracy in BVOC emissions has a large effect on determining with confidence the sources and distribution of pollutants including ozone and organic aerosols.
Accuracy in VOC measurements is important: it has been shown that even the diurnal pattern of isoprene emissions has an effect on modelling ground level ozone \cite{Hewitt_2011,Fan_2004}.

The main non-methane VOC (NMVOC) is isoprene, with emissions estimated at 500-750 Tg yr$^{-1}$ \cite{Guenther_2006}.
Isoprene is hard to directly measure, instead formaldehyde (HCHO) is often used as a proxy \cite{Marais_2012,bauwens2013satellite}.
Satellites can use DOAS techniques with radiative transfer calculations on solar radiation absorption spectra to measure column HCHO.
Several satellite data sets are publicly available and have been used to show increasing HCHO trends in developing countries \cite{Mahajan_2015}.

One estimate of global isoprene emissions is the Model of Emissions of Gases and Aerosols from Nature (MEGAN), which uses leaf area index, forest canopy modelling, and plant functional type emission factors to generate terrestrial isoprene emissions.
\citet{Guenther_2006} shows how the Australian outback is among the worlds strongest isoprene sources with large forests near coastal cities having emission factors greater than 16 mg m$^{-2}$ h$^{-1}$ (see figure \ref{fig:meganisoprene}).

Using satellite data from 1995 we have a broad measure of seasonal and medium term trends of HCHO over Australia.
These measurement can be checked against modeled estimates of isoprene using the HCHO proxy as has been done in North America using satellite and aircraft data \cite{Millet_2006}, as well as in Africa \cite{Marais_2014}.

The methodology for calculating VOCs from HCHO is laid out in \citet{Palmer_2003} and takes into account the expected lifetime and reaction rates of the precursor VOCs and HCHO.
Assuming HCHO is produced quickly from short-lived intermediates we get
\begin{eqnarray*}
VOC_i \overset{k_i}{\rightarrow} Y_i HCHO
\end{eqnarray*}
Where $Y_i$ is HCHO yields per C atom.
Then assuming a steady state between the atmospheric HCHO column ($\Omega$ molecules $cm^{-2}$)  produced by oxidation of VOCs (VOC$_i$) and no horizontal transport:
\begin{eqnarray*}
\Omega = \frac{1}{k_{HCHO}} \sum_{i} Y_i E_i
\end{eqnarray*}
Where i indexes a chemical species, and $E_i$ is emission fluxes ( C atoms $cm^{-2}s^{-1}$).

Inferring the VOC emissions then requires estimates of the HCHO yield (Y$_i$) which can be attained via modelling as done in \citet{Millet_2006} where GEOS-Chem and Master Chemical Mechanism (MCM) where both used for this.

