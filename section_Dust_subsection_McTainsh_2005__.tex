\section{Dust}

\subsection{McTainsh 2005}
4.2 Mt dust load estimated for October 2002 dust storm.

\subsection{ENSO dust \cite{Rotstayn_2011}}
Simulated enhancement of NESO-related rainfall variability due to Australian dust

CSIRO Mark 3.6 GCM is run for 160 years twice, with and without the mineral dust source in Australia set to zero.
ENSO related rainfall variability improves over eastern Australia relative to observations.


\subsection{Red Dawn: \cite{Leys_2011}}
PM10 concentrations and mass transport during "Red Dawn" - Sydney 23 September 2009.

Worst dust storm to hit Sydney since records began in 1940, with heavy dust for nine hours and visibility reduced to 400m at the airport.
Storms are common in the arid inland but rare on the east coast.
Plume dust concentration generally calculated either by simulation, or by plume volume * aerosol concentration.
$$M = AhC$$ 
where A is plume area, h is plume height, C is g/m3 of dust, M is dust load in grams.
These measurements are hampered by coarse temporal resolution of at best three hourly intervals of the Bureau of Meteorology(BoM)'s synoptic visibility observations.

Dust concentrations sourced from some stations of the DECCW AQMN.

Visibility is used to estimate total suspended particle(TSP) concentration, here a conservative predictor of PM10 c for given V is used from (Zhangbei TODO CITE).

PM10 dust transport is defined and measured with plume dust concentrations measured at various altitudes with a kite. 

Dust storm preceded by 9 dry years, 13 hot years, and was itself the hottest year on record.

PM10 normally $10{\mu}g/m^3$ in Sydney peaked at $15366{\mu}g/m^3$.
\~2.54 Mt of PM10 dust transported off shore.
Storm on par with large Chinese dust storm.
Dust ceiling around 2500m, up to 3000m.

\subsection{Atlantic Ocean Deposition \cite{Rijkenberg_2014}}
The Distribution of Dissolved Iron in the West Atlantic Ocean.

\begin{quote}
Highly accurate dissolved Fe (DFe) values measured at an unprecedented high intensity (1407 samples) along the longest full ocean depth transect (17500 kilometers) covering the entire western Atlantic Ocean.
\end{quote}

\begin{quote}
Vertical processes such as the recycling of Fe resulting from the remineralization of sinking organic matter and the removal of Fe by scavenging still dominated the distribution of DFe. In the northern West Atlantic Ocean, Fe recycling and lateral transport from the eastern tropical North Atlantic Oxygen Minimum Zone (OMZ) dominated the DFe-distribution.
\end{quote}

\subsection{Indian Ocean Deposition Measurements \cite{Heimburger_2013}}
Samples from Kerguelen and Crozet islands January - November 2010, compared against SLRS-4 and SLRS-5

Daily Fe deposition at three sites over the two islands in ${\mu}gm^{-2}d^{-1} \pm SD$:
29.1 $\pm$ 1.3, 38.7 $\pm$ 4.1, 34.6 $\pm$ 2.1

\subsection{Indian Ocean Deposition \cite{Grand_2015}}
Dust deposition in the eastern Indian Ocean: The ocean perspective from Antarctica to the Bay of Bengal.

Upper ocean distribution of dissolved Fe and Al in the eastern Indian ocean along a $95^{\circ}$E meridional transect.
Using dissolved Al measured from the CLIVAR-CO$_{2}$ program is used with a simple steady state model to produce estimates of total dust deposition. This assumes many things including that the surface water concentrations of elements are in a steady state to balance out scavenging, which allows the estimate of deposition required to maintain the steady state.

Assuming Al solubility based on latitude (mean at 3.6\%, varies from 0.1 to 49.6\%), with residence time of 5 yrs(elsewhere from 0.5 to 15 yrs), this paper uses a modified version of the Measurement of Al for Dust Calculation in Oceanic Waters(MADCOW).

Conclusion that dust deposition is not the dominant annual source of dFe in the Indian sector of the Southern Ocean, upwelling and deep mixing inputs likely are.
Does not rule out dust importance in short terms(during storms etc).

Some Results:
Minimum dust flux at Southern Ocean : 66$\pm$60 mg m$^{-2}$ yr$^{-1}$, 
Maximums for south Indian subtropical gyre: 940 mg m$^{-2}$ yr$^{-1}$ (at 18S),
southern Bay of Bengal: 2500$\pm$570 mg m$^{-2}$ yr$^{-1}$.

Annual dust deposition in the Southern ocean at around 90$^{\circ}$E is estimated to be around 7\% of the annual mean, Australian dust is mostly deposited east of 90$^{\circ}$E.

  
  
  
  