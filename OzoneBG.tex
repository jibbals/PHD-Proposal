\subsection{Tropospheric ozone}

\textbf{Ozone and PM deaths better words pls}

There is evidence for chronic effects on mortality through several large cohort studies for PM2.5 (Hoek et al 2002, Krewski et al 2009, Lepeule et al 2012), while evidence for chronic effects of ozone derives mainly from one study (Jerrett et al 2009). 

\begin{quote}
Anenberg et al (2010) used output from a global atmospheric model to estimate 3.7 $\pm$ 1.0 million deaths annually due to anthropogenic (present-day relative to preindustrial) changes in PM2.5 and 0.7 $\pm$ 0.3 million due to ozone. Brauer et al (2012) used high-resolution satellite observations of PM2.5 together with a global atmospheric model and an extensive compilation of surface measurements to better represent global air pollution exposure. These exposure estimates were then used to estimate 3.2 $\pm$ 0.4 million premature deaths due to PM2.5 and 150000 (50000 to 270000) due to ozone (Lim et al 2012).
\end{quote} \cite{Silva_2013}

Silva 2013 uses a model ensemble from ACCMIP of 14 models comparing aerosol loads from 1850(preindustrial) to 2000. Using methods as in Anenberg et al 2010 they determine premature mortality and with Monte Carlo methods determine the CIs.

Yearly global deaths from ozone, PM2.5:
472000(149/million), 2110000(665/million) \cite{Silva_2013}
\textbf{END OF BETTER WORDED SECTION}

Ozone in the lower atmosphere is a serious hazard that causes health problems \cite{Hsieh_2013}, damages agricultural crops worth billions of dollar \cite{Avnery_2011}, and increases the rate of climate warming \cite{IPCC_2013_chap8}. 
Ozone and is one of the six substances monitored by the Ambient Air Quality (AAQ) National Environment Protection Measure (NEPM), which is the Australian framework for air quality measurement and reporting aiming for ``adequate protection of human health and well-being.''
NEPM AAQ benchmarks are shown in figure \ref{fig:nepm}.

Photochemical oxidation of CO, CH4, and other non-methane volatile organic chemicals (NMVOCs) while NOx is present, as well as downward stratospheric transport, are the two main sources of tropospheric ozone \cite{Stevenson_2006}.
HCHO and NO2 regulate tropospheric oxidation capacity through O3 production, as well as being health hazards.
The HCHO/NO2 ratio can be used to determine whether surface O3 is NO2 or VOC limited \cite{Mahajan_2015}.
NO2 is a common pollutant in populated areas, released primarily by power generation and transport (TODO:cite). 

Photolysis of NO2 forms NO + O which reacts with O2 to form O3 which reacts with NO to form NO2 + O2, these reactions reach an equilibrium where O3 is proportional to the ratio between NO2 and NO.
VOCs affect the NO2 to NO ratio which changes the equilibrium and can increase O3 creation, the following formulae show an example of this with CO however similar reactions occur for VOCs:
\begin{eqnarray*}
\label{eqn:noxtoo3}
NO_2 + hv &\overset{k_1}{\rightarrow}& NO + O \\
O + O_2 &\overset{M}{\rightarrow}& O_3 \\
NO + O_3 &\overset{k_2}{\rightarrow}& NO_2 + O_2 \\
[O_3] &\sim& \frac{k_1}{k_2} * \frac{[NO_2]}{[NO]} \\
OH + CO &{\rightarrow}& HOCO \\
HOCO + O_2 &{\rightarrow}& HO_2 + CO_2 \\
HO_2 + NO &{\rightarrow}& OH + NO_2 \\
\end{eqnarray*}
The balance of these reactions is:
\begin{eqnarray*} CO + 2O_2 + hv {\rightarrow} CO_2 + O_3 \end{eqnarray*}

While photochemical production is the dominant source, stratosphere to troposphere transport (STT) of ozone is also important and climate change may drastically increase this quantity \cite{Hegglin_2009}.
The amount of ozone from STTs and photochemical production are estimated to be around 550 Tgyr-1 and 5100 Tgyr-1, respectively \cite{Stevenson_2006}. 
The main ozone removal processes are chemical destruction and wet deposition.

Ozone-rich air mixes irreversibly down from the stratosphere during certain meteorological conditions \citep{Sprenger2003,Mihalikova2012}.
In the extra-tropics, STTs most commonly occur during synoptic-scale tropopause folds \citep{Sprenger2003} and are characterised by tongues of high Potential Vorticity (PV) air descending to low altitudes.
PV is a metric which can be used to determine whether a parcel of air is stratospheric based on it's local rotation and stratification.
These tongues become elongated and filaments separate from the tongue which mix into tropospheric air.
Stratospheric ozone brought deeper (lower) into the troposphere is more likely to affect the surface ozone budget and tropospheric chemistry \citep{Zanis2003,Langford_2009}.

One dominant precursor to ozone is isoprene (C5H8), which in the atmosphere reacts rapidly with hydroxyl radicals (OH) to form peroxy radicals (HO$_2$), which react with nitrogen oxides and can lead to ground-level ozone formation similarly to the CO reaction in equation \ref{eqn:noxtoo3}.

  