\subsection{Improved GEOS-Chem resolution over Australia}
\label{resolutionwork}
\subsubsection{Preliminary Work}
A preliminary understanding of the mechanics required to both build and run the model at various resolutions has been achieved through active modelling and participation in IGC7.

\subsubsection{Going Forward}
Higher resolution modelling over Australia will allow for both improved validation with in situ measurements, and regional scale simulation analysis. 
An increase in maximum resolution from 2$^{\circ}$x2.5$^{\circ}$ to 0.25$^{\circ}$x0.3125$^{\circ}$ will be implemented.
Dust simulation in particular can be greatly improved with finer surface wind resolution.

During this PhD the GEOS-Chem model will be improved by allowing enhanced horizontal resolution over Australia and possibly NZ.
This has been done already over China \cite{Chen_2009,Wang_2004}, North America \cite{Zhang_2012}, Europe \cite{Protonotariou_2013}, and the Amazon \cite{Barkley_2013}.

After determining the desired spatial and temporal boundaries for improved resolution, raw data will be downloaded, verified, and re-gridded using existing code (developed for other regions by the GEOS-Chem support team).
To ensure that we have an acceptable inventory for Australian emissions existing inventories from state EPAs may need to be patched together.

Finally, emissions data needs to be reworked over the domain, ensuring emissions factors are maintained.
This last requirement may be eased by the new HEMCO module implemented within the ESMF environment (compatible with GEOS-Chem v10) which recalculates emissions on any user specified grid.
  