\subsection{Improved GEOS-Chem Modelling resolution}

Higher resolution modelling over Australia will allow for both improved validation with in situ measurements, and regional scale simulation analysis. 
Dust simulation in particular can be greatly improved with finer surface wind resolution and will be one of the areas I will focus on with validation against both in situ and satellite measurements of AOD and dust concentrations across Australia.
One of my goals for my PHD will be to improve the GEOS-Chem model by setting up meteorological fields with higher horizontal resolution over Australia and NZ.
This has been done already over China \cite{Chen_2009}, North America, and Europe ( 2 more cites needed) but not for anywhere in the southern hemisphere.

First sufficient disk space will be required to hold several years worth of high resolution meteorological data and emissions, this is expected to be to the order of 100s of gigabytes.
After determining the desired spatial and temporal boundaries for improved resolution, raw data will be downloaded, verified, and re-gridded using partially written code from Bob Yantosca.
Finally emissions data needs to be reworked over the domain, ensuring emissions factors and chemistry is maintained.
This last requirement may be eased by the new HEMCO module implemented within the ESMF environment (compatible with GEOS-Chem v10) which recalculates emissions on any user specified grid.
  
  