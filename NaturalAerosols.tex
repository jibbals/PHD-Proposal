\subsection{Natural gas and aerosol emissions in Australia}

Australia is largely covered by environments which are not heavily influenced by human interactions.
These environments are sources of naturally released trace gases which make up some small portion of the air we breathe.
Naturally occurring trace elements in the atmosphere can have a large impact on living conditions and react in complex ways with anthropogenic emissions as well as affecting various ecosystems upon which life depends.
These biogenics affect surface pollution levels and can alter the radiative and particulate matter distribution of the atmosphere with harmful results.
For example; ozone in the lower atmosphere is a serious hazard that causes health problems \cite{Hsieh_2013}, damages agricultural crops worth billions of dollar \cite{Avnery_2011}, and increases the rate of climate warming \cite{IPCC_2013_chap8}.

The Australian outback includes extremely large and diverse environments.
Much of the landscape outside of urban areas is undeveloped and sparsely inhapbited.
In Australia most air quality measurements are performed in cities and trace gas readings are influenced greatly by anthropogenic sources.
Estimates of atmospheric gas and particulate densities and distributions over the rest of the continent are uncertain due to a lack of in-situ measurements.
  