\subsection{Dust}
\subsubsection{Preliminary Work}
Australian dust emissions are largely estimated using global models which are not tuned for Australian conditions \cite{Ridley_2013,Duncan_Fairlie_2007}.

Using GEOS-Chem model version 9.02 and comparing against AERONET and CoDii in situ data allows a model validation in an area which is largely estimated due to lack of ground based measurements.

Running GEOS-Chem at 2x2.5 resolution, using offline GEOS-5 meteorological fields and DEAD dust mobilisation provides a simulation of global dust emission, deposition, and transport.
DEAD dust mobilisation is based on surface wind speeds to the third power, implemented by \citet{Duncan_Fairlie_2007}.
We only model emission, deposition, and transport of dust and carbon, allowing a fast runtime.
Running the full chemical model to see if dust was affected by other tracers found only negligible differences (in the order of 10$^{-5}$ percent).
With 2004 as the spin up year, monthly average columns of sources and sinks and AOD are simulated until November 2012.

There are very few in-situ measurement stations recording AOD which are predominantly dust related.
Two which do exist are within AERONET, Tinga Tingana and Birdsville, and a direct comparison between these sites can be seen in figure \ref{fig:AODComparison}.
These records of AOD show a fairly close match to the model, with much of the observed variability captured, especially at Birdsville.
